\documentclass{article}
\usepackage{amsmath}
\usepackage{mathtext}
\usepackage[english,russian]{babel}
\usepackage[T2A]{fontenc}
\usepackage[utf8]{inputenc}
\usepackage[dvips]{graphicx}
\graphicspath{{img/}}
\DeclareGraphicsExtensions{.jpg, .png, .pdf}
\begin{document}
\section{Hello, world!!}
\section{Введение, условие эксплуатации}

Зачем надо:
\begin{itemize}
	\item Круто
	\item Полезно
	\item Восхитительно
\end{itemize}

\subsection{Компоненты сетей}

Сети включают в себя:
\begin{itemize}
	\item Устройства
	\item Средства подключения
	\item Сервисы
\end{itemize}
% Картинка

Устройства соединенные сетью можно разделить на:
\begin{itemize}
	\item Оконечные (пользовательские) устройства
	\item Промежуточные устройства
\end{itemize}

Задача сетевого инженера состоит в том чтобы соединить эти устройства оптимальным вариантом
согласно определенным критериям.

\subsection{Сетевые устройства}
% Картинка

\underline{Концентратор} 

\underline{Коммутатор} -- Способен обрабатывать информацию

\underline{Мост} -- На программном уровне разделяет 2 домена коллизий

\underline{Коммутатор 3го уровня} -- Обрабатывает ip пакеты, которые ходят по 3го уровню ip протокола

\subsection{Интерфейс}

\underline{Интерфейс} -- некоторая физическая или программная сущность взаимодействия устройства с сетью или с другими устройствами.

\underline{Физический порт} -- стандартизированный разъем на сетевом устройстве, для подключения к нему кабелей.

\underline{Протокол} -- набор правил и действий (очерёдности действий), позволяющий осуществлять соединение и обмен данными между двумя и более включёнными в сеть устройствами.

\subsection{Сетевая модель}

Сетевая модель, или сетевая структура, представляет собой упорядоченный набор документации и стандартов, которому должны соответствовать все устройства в сети для обеспечения ее работоспособности.

TCP/IP -- Transmission Control Protocol / Internet Protocol
OSI -- Open System Interconnection

RFC -- Requsts for Comments -- ресурс для новых протоколов
IEEE -- Institute of Electrical and Electronic Engeneers -- Одна из организация занимающихся стандартизацией


\end{document}
