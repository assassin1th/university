\documentclass{article}
\usepackage{amsmath}
\usepackage{mathtext}
\usepackage[english,russian]{babel}
\usepackage[T2A]{fontenc}
\usepackage[utf8]{inputenc}
\usepackage[dvips]{graphicx}
\graphicspath{{img/}}
\DeclareGraphicsExtensions{.jpg, .png, .pdf}
\begin{document}
\section{что-то}
\subsection{Параметры, характеризующие логические и схемотехнические возможности логических элементов БИС и СБИС}
\begin{enumerate}
	\item Реализуемая логическая функция
	\item Нагрузочная спосбоность $n$.
Логические элементы КМОП обладают крайне высоким входным сопротивлением.
В статическом режиме практически не отдают ток в нагрузку и не потребляют оного.
Поэтому нагрузочная способность высока $> 10$
Но ее увеличение ведет к ухудшению динамических характеристик, из-за паразитной емкости (рост постоянных времени разряда и заряда)
Увеличивается вермя разрядки и зарядки.
	\item Коэффициент объединения по входу $m$. Характеризует максимальное число входов ЛЭ (логических элементов).
	С увеличением параметра $m$ увеличивается функциональные возможности ЛЭ за счет возможности выполнения функции с бОльшим числом аргументов
	на одном типовом элементе. Однако, при увеличении числа входов, как правило, ухудшаеются другие параметры ЛЭ.
	С точка зрения возможности увеличения коэф. объединения по входу схемы реализующие функцию И и функцию ИЛИ существенно отличаются друг от друга.
	
	В ЛЭ где функция первой ступения выполняется на транзисторах требуется для увеличения $m$ требуется значительный рост размеров кристалла и кол-ва
	компонентов. В особенности это относится к КМОП.
	\item Среднее время задержки передачи сигнала $\tau_{ср}$
	%Картинка c сигналом

	Если на вход подать трапециидальный импульс, то на выходе мы получем смещенный импульс на велчину смещения -- $t_{зд}$. Данная ситуация имеет следующие параметры
	\begin{itemize}
		\item Среднее время задержки распростронения сигнала -- динамический параметр ЛЭ. Его определение обеспечивается сравнением входного и выходного сигнала
		$t_{зд. р. ср.} = \frac{t_{зд}^{1,0} + t_{зд}^{0,1}}{2}$
		\item $t_{зд}^{1,0}$ -- время перехода из состояния единицы в в логический ноль. (точка перехода составляет 90\% от сигнала)
		\item $t_{зд}^{0,1}$ -- время перехода сигнала из нуля в логическую единицу
		\item $t^{1,0}$ -- время нисходящего фронта. логический ноль составляет величину равную 10\% от величины сигнала.
	\end{itemize}
	Среднее время распростронения сигнала служит усредненным параметром оценки быстродействия используемым при расчете временных характеристик многоэлементных
	последовательно-включенных логических микросхем.

	Параметр приводится во всех технических условиях или руководящих материалов по применению этих интегральных микросхем.
	\item Предельная рабочая частота $f_{пр}$. Это предельная частота переключения асинхронного RS-триггера реализованного на базе этого типа логических
	элементов. 
	\item Помехоустойчивость.
	%Картинка
	ЛЭ в статическом режиме может находиться в одном из двух устойчивых состояний (1 или 0). По этой причине различают статическую помехоустойчивость по уровню логического нуля
	и по уровню логической единицы. Статитеческая помехоустойчивость ЛЭ опредлеяется значением напряжения, которое может быть подано на его вход относительно уровня логического
	нуля или уровняю логической единицы, не вызывая его ложного срабатываения.

	Значение статической помехоустойчивости по уровню лог 0 и лог 1 определяют из анализа семейства передаточных характеристик ЛЭ.
	%Серия картинок с шагами определения
	A -- Рабочая точка при уровне лог 1\\
	B -- Рабочая точка при уровне лог 0\\
	С -- Неустойчивая рабочая точка\\
	Проводим касательные к изначальной характеристике под углом $45^{\circ}$
	Таким образом находим запасы по уровню логической единицы -- $U_{п1}$ и по уровню логического нуля -- $U_{п0}$
	%Еще картика

	Запас помехоустойчивости дается для наихудшего случая
	
	Динамическая помехоустойчивость ЛЭ -- как ЛЭ реагирует на динамические помехи. В общем случае зависит от длительности, мощности и формы сигнала помехи,
	а также от уровня статической помехоустойчивости и быстродействия
	%Еще картинка
	$\tau_{k_1}$ - длительность помехи при которой помехоустойчивость больше не зависит от амплитуды помехи, а только от ее длительности.
	\item Потребляемая мощность $P_{ср}$. При работе в реальном устройстве. Каждый логический элемент может находиться в следующих состояниях:
	\begin{enumerate}
		\item Состояние выключено.
		\item Стадия влючения.
		\item Состояние включено.
		\item Стадия выключения.
	\end{enumerate}
	В каждом из перечисленных состояний логический элемент потребляет от источника различную мощность. Средняя потребляемая мощность определяется как
	полусумма мощностей в каждом из состояний.

	КМОП схемы не потребляют мощность в состояниях включено выключено, но потребляют мощность при переходах. Для таких схем вводят параметр динамической потребляемой мощности.
	Она пропорциональная квадрату напряжения умноженная на частоту переключения.
	\item Энергия переключения $Э_{пер} = P_{ср} \times \tau_{ср}$. Характеризует качество разработки и исполнения логического элемента. Чем меньше этот параметр, тем выше качество разработки.
	Для большинства семейств цифровых микросхем $0.1 \le Э_{пер} \le 500 пДж$.
\end{enumerate}
\end{document}
