\documentclass{article}
\usepackage{amsmath}
\usepackage{mathtext}
\usepackage[english,russian]{babel}
\usepackage[T2A]{fontenc}
\usepackage[utf8]{inputenc}
\begin{document}
%Тут должен быть рисунок
Линия зацепления $N_1N_2$ --- тракетория общей точки контакта $K$ зубьев при вращении колес при передаче.

$N_1N_2$ --- теоретическая линия зацепления, ее обозначают $g$

В реальности контакт между зубьями будет на отрезке $AB$ называемый активной линией зацепления.

$\varphi_{\gamma}$ - угол перекрытия зубчатых колес.

Угол перекрытия показывает на каком углу колесо входит в зацепление и выходит из него.
Для более плавной и качественной передачи момента коэфициент перекрытия должен быть больше $1$, обычно $1.2$.
$\alpha$ --- угол зацепления (угол между коризонтальной прямой и теоритической линией зацепления).
$$
AB = g \alpha
$$
$$
\varepsilon_{\gamma}=\frac{\varphi_{\gamma}}{\tau}
$$
Межосевое расстояние $a$ это расстоянием между осями зубчатых колес. наиболее часто применяют зубчатые колеса с так называемым делительным межосевым расстоянием. При этом в полюсе контакта касательными являются делительные окружности.
В данном случае они же и являются начальными (центроидами)
$$
a = r_1 + r_2 = \frac{mz_1}{2} + \frac{mz_2}{2}
$$
Анализ сил и моментов в одноступенчатой зубчатой передаче.
%Здесь должна быть картинка
$$
F_{n_1} = F_{n_2}
$$
$$
F_{n_1} = \frac{2m_1}{d_1}
$$
$$
F_{n_2} = \frac{2m_2}{d_2} \to \frac{m_2}{m_1} = \frac{d_2}{d_1}=i_{12}
$$
Расчет модуля зубчатых колес
%Здесь должна быть еще одна картинка

$F_{n} \cos{\alpha}$ - полезная сила

$F_{n} \sin{\alpha}$ - бесполезная сила  

Так как материалы при сжатии выдерживают нагрузку намного больше чем при растяжении, под опасной точкой будем понимать точку $A$.
$$
F_n = \frac{2M_2}{mz_2\cos{\alpha}} = \frac{2M_1}{mz_1\cos{\alpha}}
$$
$$
\sigma_u = 
$$
$$
\sigma_p = \frac{\sigma_{пред}}{n_{т}}
$$
$$
\sigma_{пред} \to \sigma_{т} \sigma_{В} \sigma_{пп}
$$
$$
n_т = 1.5
$$
$$
\sigma_\Sigma = \sigma_{cж} - \sigma_{и}
$$
$$
\sigma_{сж} = \frac{f\sin{\alpha}}{bs}
$$
$$
b = \Psi_{вм}m
$$
$$
\Psi_{вм} = \frac{b}{m}
$$
$$
\Psi_{вм} = 3..16
$$
$$
W_{изг} = \frac{bs^2}{6}
$$
$$
M_{изг} = h_рF_n\cos{\alpha}
$$
$$
\sigma_и = \frac{6h_рF_n\cos{\alpha}}{bs^2}
$$
$$
\sigma_р = \frac{F_n\sin{\alpha}}{bs}
$$
$$
\sigma_{\Sigma} = \frac{6h_рF_n\cos{\alpha}}{bs^2} - \frac{F_n\sin{\alpha}}{bs} = \frac{F_n}{b}(\frac{6h_р\cos{\alpha}}{s^2} - \frac{F_n\sin{\alpha}}{s})
$$
$$
F_n = \frac{2M_2K}{mz_2\cos{\alpha}}
$$
$$
K = K_bK_V
$$

Где $K_b$ - коэфициент концентричности напряжения
$K_V$ - кожфициент динамичности нагрузки
%Еще много всяких формул (жуть)

Основной формулой для расчета контактных напряжений является формула Герца для контакта 2х целинтров:

$$
\sigma_M=\sqrt[2]{\frac{E_прq_n}{\rho_пр(1-\mu^2)2\pi}}
$$
$$
E_{пр} = \frac{2E_1E_2}{E_1 + E_2}
$$
$$
\rho_{пр} = \frac{\rho_1\rho_2}{\rho_1 + \rho_2} = \frac{1 + i_{12}}{r_1i_{12}}\\%дописа
$$
$$
q_n = \frac{F_n}{b}
$$
$$
d_2 = 2r_2 = 2i_{12}r_1 = \frac {2 a i_{12}}{i_{12} + 1}
$$
$$
q_n = \frac {2M_2 K}{d_2 b \cos{\alpha}} = \frac{2 M_2 k (i_{12} + 1)}{b \cos{\alpha} 2 a i_{12}}
$$
$$
\sigma_M = \sqrt[2]{\frac{2 E_1 E_2 2 M_2 K (i_{12} + 1) (1 + i_{12})}{(E_1 + E_2) b \cos{\alpha} 2 a i_{12} a i_{12} \sin{\alpha} 2 \pi (1 - \mu^2)}}
$$

Приводя некоторые сокращения:

$$
\sigma_M = \sqrt[2]{\frac{2 E_1 E_2 2 M_2 K (1 + i_{12})^3}{(E_1 + E_2) \psi_{ba} \sin {2\alpha} a^3 i_{12}^2 2 \pi (1 - \mu^2)}}
$$
$$
z_M = \sqrt[2]{\frac{2}{\sin{2\alpha}}}
$$
$$
z_m = \sqrt[2]{\frac{E_{пр}}{2\pi(1-\mu^2)}}
$$

Таким образом:
$$
\sigma_M = z_M z_m z_\varepsilon - \sqrt[2]{\frac{k M_2 ( 1 - i_{12})^3}{\Psi_b a a^3 i_{12}^2}}
$$

Тогда:

$$
a = (i_{12} + 1)\sqrt[3]{\frac{K M_2}{\Psi_ba}(\frac{z_m z_M z_\varepsilon}{i_{12}[\sigma_M]})^2}
$$

Если рассечь косозубое колесо перпендикулярно оси зубьев, то в сечении будет эллипс и эвольвентный профиль зубьев

%картинка

$F_n$ -- нормальная сила
$F$ -- окружная сила
$F_a$ -- осевая сила
$F_r$ -- радиальная сила
Для той же точки но только на проекции зуба

% Еще одна картинка

\begin{enumerate}
	\item $F = \frac{2 M_2}{d_2} = \frac{2 M_2}{m_t z_2} = \frac{2 M_2 \cos{\beta}}{m_n s_2}$
	\item $F'_n = \frac{F}{ \cos{\beta}} = \frac{ 2M_2}{m_n z_2} \frac{ \cos{\beta}}{ \cos{ \beta}} = \frac{2 M_2}{ m_n z_2}$
	\item $F_a = f \tan{ \beta} = \frac{2 M_2 \cos{ \beta} \tan{ \beta}}{m_n z _2} = \frac{2 m_2 \sin{ \beta}}{m_n z_2} $
	\item $F_n = \frac{F'_n}{ \cos{ \alpha}} = \frac{F}{ \cos{ \alpha} \cos{ \beta}} = \frac{ 2 M_2}{m_n z_2 \cos{ \alpha}} $
\end{enumerate}

В домашнем задании необходимо:
\begin{enumerate}
	\item Рассчитать отсчетное устройство
	\item подобрать двигатель
\end{enumerate}

Методичка -- расчет электромеханического привода (ЭМП) кокорев юрий алексеевич

Расчетное устройство

Задача:

\begin{itemize}
	\item Выбрать и рассчитать отсчетное устройство предназначенное для зрительного определения вводимой величины, изменяющейся в интервале: $x_{max} - x_{min} = 60 о. е.$ о. е. -- отсчетные единицы $ \delta x = 0.01 o. e.$ Расчетное должно быть с равномерной отстчетной шкалой. Наблюдение происходит при номральном освещении с расстояния $250-400мм$
\end{itemize}

\begin{enumerate}
	\item Отсчетные устройства бывают:
	\begin{enumerate}
		\item с неподвижной шкалой и подвижным указателем. Для этих устройств характерно: снятие показаний
		\item с подвижной шкалой и неподвижным указателем. Для этих устройств хаарктерно: ввод информации
	\end{enumerate}
	\item Определение цены деления шкалы
	$$
	b = H = 2 \Delta x
	$$
	$H = | 2 \Delta x| = 2 * 0.01 = 0.02 о. е.$
	\item Определение числа делений шкалы $N = \frac{x_{max} - x_{min}}{H} = \frac{60}{0.02} = 3000$
	\item Определение длины деления шкалы. Для зрительного определения при нормальных условиях с расстояния $250-400 мм$ рекомендуется брать расстояние между штрихами $b = (1..2.5) мм$. Пусть $b = 2 мм$, тогда $L = 3000 * 2 = 6000 мм$ 
	\begin{enumerate}
		\item Отсчетное устройство с прямолинейной шкалой (горизонтальной или вертикальной) $L = N b = 3000 * 2 = 6 м$ такой вариант нам не подходит
		\item Отсчетное устройство с круговой или цилиндрической шкалой $L = \pi D \frac{ \psi}{360} $. $ \psi$ -- угол градуировки
		$$
		D = \frac{L 360}{ \pi \psi}  = \frac{6000 * 360}{3.14 * 360} 
		$$
		Данный тип расчетного устройства не подходит
		\item Двухшкальные отсчетные устройства $N_{што}$ -- количесвто делений шкалы точного отсчета -- назначаются из стандартного ряда [10, 20, 50, 100]. $N_{шго}$ -- количество делений шкалы грубого отсчета -- вычислияются не округляется. Обязательно должно выполняться условие:
		$$
		N_{што} > N_{шго}
		$$
		$$
		N_{што} = \frac{N}{N_{шго}} 
		$$
		$$
		H_{што} = H
		$$
		$$
		H_{шго} = H_{што} N_{што} = 2 о. е.
		$$
		Таким образом длина счетной шкалы:
		$$
		L_{што} = N_{што} b = 200мм
		$$
		$$
		L_{шго} = N_{шго} b = 60 мм
		$$
		Тогда диаметры счетных дисков:
		$$
		D_{што} = \frac{L_{што}}{ \pi} \frac{360}{ \psi} = 64 мм
		$$
		$$
		D_{шго} = \frac{L_{шго}}{\pi} \frac{360}{ \psi} = 19 мм
		$$
		Необходимо брать диаметры шкал из ряда [35, 50, 65, 80, 100] мм. В данном случае мы выбираем диаметр 65мм. Тогда нам необходимо пересчитать $b$:
		$$
		b_{што} = \frac{D_{што} \pi \psi_{што}}{360 N_{што}} =  2.04 мм\label{b1}
		$$

		$$\label{b2}
		b_{шго} = \frac{D_{шго} \pi \psi_{шго}}{360 N_{шго}} = 6.8 мм
		$$
		В \ref{b1} мы должны получить число от 1 до 2.5. В то время как \ref{b2} мы не трогаем.
		$$
		i_{што-шго} = \frac{\varphi_{што}}{\varphi_{шго}} = \frac{N_{шго} 360}{ \psi}  = 30
		$$
		При разработке механизма отсчетного устройства обычно связывают. Шкалу грубого отсчета с выходным валом, а шкалу точного отсчета с любым предыдущим валом, но при выполнении ряда условий.
		\begin{enumerate}
			\item И ШТО и ШГО должно быть равно расчетной величине
			\item ШГО должна располагаться слева от ШТО со стороны наблюдателя
			\item Шкалы должны вращаться в одну сторону
			\item Шкалы должны быть расположены горизонтально
		\end{enumerate}
		Какие условия нужно будет соблюзти при проектировке
		\begin{enumerate}
			\item И ШТО и ШГО должны быть соот друг другу
			\item ШГО слева от ШТО
			\item ШТО и ШГО должны распологаться на одной оси
		\end{enumerate}
		$$
		x_{max} - x_{min} = \varphi
		$$
	\end{enumerate}
\end{enumerate}

Расчет ЭМП

ЭМП --- (электромеханический привод) --- это устройство состоящее из двух основных частей%тут картинка
\begin{itemize}
	\item Электродвигателя, осуществляющего преобразование электрической энергии в мехнаническую
	\item Редуктор, связывающего электродвигатель с рабочим органом (нагрузкой)
\end{itemize}

Электродвигатели бывают:
\begin{enumerate}
	\item По назначению:
	\begin{enumerate}
		\item Общего назначения. Для работы в нерегулируемом приводе
		\item Исполнительный. Для работы в следящих системах
	\end{enumerate}
	\item По приципу действия:
	\begin{enumerate}
		\item Постоянного тока
		\item Переменного тока:
		\begin{enumerate}
			\item Синхронные
			\item Асинхронные
		\end{enumerate}
		\item Универсальные
		\item Шаговые
	\end{enumerate}
\end{enumerate}

Для того чтобы правильно выбрать электродвигатель необходимо знать:
\begin{enumerate}
	\item Тип привода и режимы его работы (следящий и нерегулируемый привод)
	\item Тип источника питания и его характеристики:
	\begin{itemize}
		\item Для источников постоянного тока необзодимо знать напряжение и ток
		\item Для источников переменного тока (необходимо знать напряжение частоту и ток)
	\end{itemize}
	\item Характеристику нагрузки:
	\begin{enumerate}
		\item Максимальная величина момента
		\item Номинальная углованя скорость вращения
		\item Момент инерции нагрузки
		\item Требуемое быстродействие
	\end{enumerate}
	\item Эксплуатационные условия
	\begin{enumerate}
		\item Окружающая среда (условия работы привода)
		\item Ресурс работы (время жизни в часа)
		\item Массо-габаритные параметры
	\end{enumerate}
\end{enumerate}

Пример: Необходимо выбрать электродвигатель для нерегулируемого нереверсивного привода при следующих исходных данных:
\begin{enumerate}
	\item $M_н = 0.5 н * м$
	\item $J_н = 0.1 кг м^2$
	\item $\varepsilon_н = 10 \frac{рад}{с^2}$
	\item $W_н = 2 \frac{рад}{с} $
	\item $L = 5000 ч$
	\item $t_p = 0.1 с$
	\item $U = 36 В$
\end{enumerate}

Для того чтобы определить двигатель необходимо сначала расчитать требуюмую мощность:
$$
P_н = N_н \omega_н
$$
$$
P_р = \frac{P_н}{\eta} = \frac{N_н \omega_н}{ \eta} = 1.25 Вт
$$

Мы берем кпд равным 0.8


$\xi = (1.2 .. 2.5)$ -- обычная точность
$\xi = (2.5 .. 5)$ -- повышенная точность

$$
1.25 * 1.2 \le P_t \le 1.25 * 2.5
$$
$$
1.5 Вт \le P_t \le 3.125
$$

мы выбрали двигатель Дид-3т по какой-то очень важной причине. Его параметры:
\begin{itemize}
	\item $P_н = 3.6 Вт$
	\item $M_{ном} = 5.6 н * мм$
	\item $n_{ном} = 8000 \frac{об}{мин}$
	\item $M_п = 100 н * мм$
	\item $J_р = 24 * 10^{-8} кг * м^2$
	\item $Т_{эм} = 26 мс$
	\item $U = 36 В$
	\item $Масса = 0.35 кг$
\end{itemize}

Проверка двигателя по моментам. Для нерегулируемого нереврсивного привода характерны:
\begin{itemize}
	\item Продолжительный ресурс работы
	\item Редкие пуски
	\item Отсутсвие реверсов
\end{itemize}
$$
M_п \ge M_\Sigma
$$
$$
M_н \ge M_{стпр}
$$
где:
$$
M_{стрп} = \frac{М_н}{\eta i_0} 
$$
$$
M_{дим пр} = J_{пр} \varepsilon_н i_0 = [(1 + K_м)J_р + \frac{J_н}{i_0^2} ]\varesilon_н i_0
$$
$$
\omega_{дв} = \frac{n_{ном} \pi}{30}
$$
Таким образом передаточный коэфициент:
$$
i_0 = \frac{ \omega_{дв}}{ \omega_н} = \frac{n_{ном} \pi}{ \omega_н 30} = 420
$$
$$
M_{стрп} = 1.5 Н мм
$$
Приведенный момент инерции считать надо отдельно

$$
J_{пр} = (1 + K_М)J_р + \frac{J_н}{i_0^2} = 104.6 * 10^{-8} кг * м^2
$$

$K_М$ -- коэфициент учитывающий инерционные параметры собственного зубчатого механизма

Подставляем момент динамический приведенный
$$
M_{дин пр} = J_{пр} * \varepsilon_н * i_0 = 4.4 Нмм
$$

Так как условию %надо сослаться на условие
удовлетворяет проверка по моментам пройдена
\end{document}
