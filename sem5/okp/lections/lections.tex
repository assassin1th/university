\documentclass{article}
\usepackage{amsmath}
\usepackage{mathtext}
\usepackage[english,russian]{babel}
\usepackage[T2A]{fontenc}
\usepackage[utf8]{inputenc}
\begin{document}
%Тут должен быть рисунок
Линия зацепления $N_1N_2$ --- тракетория общей точки контакта $K$ зубьев при вращении колес при передаче.

$N_1N_2$ --- теоретическая линия зацепления, ее обозначают $g$

В реальности контакт между зубьями будет на отрезке $AB$ называемый активной линией зацепления.

$\varphi_{\gamma}$ - угол перекрытия зубчатых колес.

Угол перекрытия показывает на каком углу колесо входит в зацепление и выходит из него.
Для более плавной и качественной передачи момента коэфициент перекрытия должен быть больше $1$, обычно $1.2$.
$\alpha$ --- угол зацепления (угол между коризонтальной прямой и теоритической линией зацепления).
$$
AB = g \alpha
$$
$$
\varepsilon_{\gamma}=\frac{\varphi_{\gamma}}{\tau}
$$
Межосевое расстояние $a$ это расстоянием между осями зубчатых колес. наиболее часто применяют зубчатые колеса с так называемым делительным межосевым расстоянием. При этом в полюсе контакта касательными являются делительные окружности.
В данном случае они же и являются начальными (центроидами)
$$
a = r_1 + r_2 = \frac{mz_1}{2} + \frac{mz_2}{2}
$$
Анализ сил и моментов в одноступенчатой зубчатой передаче.
%Здесь должна быть картинка
$$
F_{n_1} = F_{n_2}
$$
$$
F_{n_1} = \frac{2m_1}{d_1}
$$
$$
F_{n_2} = \frac{2m_2}{d_2} \to \frac{m_2}{m_1} = \frac{d_2}{d_1}=i_{12}
$$
Расчет модуля зубчатых колес
%Здесь должна быть еще одна картинка

$F_{n} \cos{\alpha}$ - полезная сила

$F_{n} \sin{\alpha}$ - бесполезная сила  

Так как материалы при сжатии выдерживают нагрузку намного больше чем при растяжении, под опасной точкой будем понимать точку $A$.
$$
F_n = \frac{2M_2}{mz_2\cos{\alpha}} = \frac{2M_1}{mz_1\cos{\alpha}}
$$
$$
\sigma_u = 
$$
$$
\sigma_p = \frac{\sigma_{пред}}{n_{т}}
$$
$$
\sigma_{пред} \to \sigma_{т} \sigma_{В} \sigma_{пп}
$$
$$
n_т = 1.5
$$
$$
\sigma_\Sigma = \sigma_{cж} - \sigma_{и}
$$
$$
\sigma_{сж} = \frac{f\sin{\alpha}}{bs}
$$
$$
b = \Psi_{вм}m
$$
$$
\Psi_{вм} = \frac{b}{m}
$$
$$
\Psi_{вм} = 3..16
$$
$$
W_{изг} = \frac{bs^2}{6}
$$
$$
M_{изг} = h_рF_n\cos{\alpha}
$$
$$
\sigma_и = \frac{6h_рF_n\cos{\alpha}}{bs^2}
$$
$$
\sigma_р = \frac{F_n\sin{\alpha}}{bs}
$$
$$
\sigma_{\Sigma} = \frac{6h_рF_n\cos{\alpha}}{bs^2} - \frac{F_n\sin{\alpha}}{bs} = \frac{F_n}{b}(\frac{6h_р\cos{\alpha}}{s^2} - \frac{F_n\sin{\alpha}}{s})
$$
$$
F_n = \frac{2M_2K}{mz_2\cos{\alpha}}
$$
$$
K = K_bK_V
$$

Где $K_b$ - коэфициент концентричности напряжения
$K_V$ - кожфициент динамичности нагрузки
%Еще много всяких формул (жуть)

Основной формулой для расчета контактных напряжений является формула Герца для контакта 2х целинтров:

$$
\sigma_M=\sqrt[2]{\frac{E_прq_n}{\rho_пр(1-\mu^2)2\pi}}
$$
$$
E_{пр} = \frac{2E_1E_2}{E_1 + E_2}
$$
$$
\rho_{пр} = \frac{\rho_1\rho_2}{\rho_1 + \rho_2} = \frac{1 + i_{12}}{r_1i_{12}}\\%дописа
$$
$$
q_n = \frac{F_n}{b}
$$
$$
d_2 = 2r_2 = 2i_{12}r_1 = \frac {2 a i_{12}}{i_{12} + 1}
$$
$$
q_n = \frac {2M_2 K}{d_2 b \cos{\alpha}} = \frac{2 M_2 k (i_{12} + 1)}{b \cos{\alpha} 2 a i_{12}}
$$
$$
\sigma_M = \sqrt[2]{\frac{2 E_1 E_2 2 M_2 K (i_{12} + 1) (1 + i_{12})}{(E_1 + E_2) b \cos{\alpha} 2 a i_{12} a i_{12} \sin{\alpha} 2 \pi (1 - \mu^2)}}
$$

Приводя некоторые сокращения:

$$
\sigma_M = \sqrt[2]{\frac{2 E_1 E_2 2 M_2 K (1 + i_{12})^3}{(E_1 + E_2) \psi_{ba} \sin {2\alpha} a^3 i_{12}^2 2 \pi (1 - \mu^2)}}
$$
$$
z_M = \sqrt[2]{\frac{2}{\sin{2\alpha}}}
$$
$$
z_m = \sqrt[2]{\frac{E_{пр}}{2\pi(1-\mu^2)}}
$$

Таким образом:
$$
\sigma_M = z_M z_m z_\varepsilon - \sqrt[2]{\frac{k M_2 ( 1 - i_{12})^3}{\Psi_b a a^3 i_{12}^2}}
$$

Тогда:

$$
a = (i_{12} + 1)\sqrt[3]{\frac{K M_2}{\Psi_ba}(\frac{z_m z_M z_\varepsilon}{i_{12}[\sigma_M]})^2}
$$

Если рассечь косозубое колесо перпендикулярно оси зубьев, то в сечении будет эллипс и эвольвентный профиль зубьев

%картинка

$F_n$ -- нормальная сила
$F$ -- окружная сила
$F_a$ -- осевая сила
$F_r$ -- радиальная сила
Для той же точки но только на проекции зуба

% Еще одна картинка

\begin{enumerate}
	\item $F = \frac{2 M_2}{d_2} = \frac{2 M_2}{m_t z_2} = \frac{2 M_2 \cos{\beta}}{m_n s_2}$
	\item $F'_n = \frac{F}{ \cos{\beta}} = \frac{ 2M_2}{m_n z_2} \frac{ \cos{\beta}}{ \cos{ \beta}} = \frac{2 M_2}{ m_n z_2}$
	\item $F_a = f \tan{ \beta} = \frac{2 M_2 \cos{ \beta} \tan{ \beta}}{m_n z _2} = \frac{2 m_2 \sin{ \beta}}{m_n z_2} $
	\item $F_n = \frac{F'_n}{ \cos{ \alpha}} = \frac{F}{ \cos{ \alpha} \cos{ \beta}} = \frac{ 2 M_2}{m_n z_2 \cos{ \alpha}} $
\end{enumerate}

В домашнем задании необходимо:
\begin{enumerate}
	\item Рассчитать отсчетное устройство
	\item подобрать двигатель
\end{enumerate}

Методичка -- расчет электромеханического привода (ЭМП) кокорев юрий алексеевич

Расчетное устройство

Задача:

\begin{itemize}
	\item Выбрать и рассчитать отсчетное устройство предназначенное для зрительного определения вводимой величины, изменяющейся в интервале: $x_{max} - x_{min} = 60 о. е.$ о. е. -- отсчетные единицы $ \delta x = 0.01 o. e.$ Расчетное должно быть с равномерной отстчетной шкалой. Наблюдение происходит при номральном освещении с расстояния $250-400мм$
\end{itemize}

\begin{enumerate}
	\item Отсчетные устройства бывают:
	\begin{enumerate}
		\item с неподвижной шкалой и подвижным указателем. Для этих устройств характерно: снятие показаний
		\item с подвижной шкалой и неподвижным указателем. Для этих устройств хаарктерно: ввод информации
	\end{enumerate}
	\item Определение цены деления шкалы
	$$
	b = H = 2 \Delta x
	$$
	$H = | 2 \Delta x| = 2 * 0.01 = 0.02 о. е.$
	\item Определение числа делений шкалы $N = \frac{x_{max} - x_{min}}{H} = \frac{60}{0.02} = 3000$
	\item Определение длины деления шкалы. Для зрительного определения при нормальных условиях с расстояния $250-400 мм$ рекомендуется брать расстояние между штрихами $b = (1..2.5) мм$. Пусть $b = 2 мм$, тогда $L = 3000 * 2 = 6000 мм$ 
	\begin{enumerate}
		\item Отсчетное устройство с прямолинейной шкалой (горизонтальной или вертикальной) $L = N b = 3000 * 2 = 6 м$ такой вариант нам не подходит
		\item Отсчетное устройство с круговой или цилиндрической шкалой $L = \pi D \frac{ \psi}{360} $. $ \psi$ -- угол градуировки
		$$
		D = \frac{L 360}{ \pi \psi}  = \frac{6000 * 360}{3.14 * 360} 
		$$
		Данный тип расчетного устройства не подходит
		\item Двухшкальные отсчетные устройства $N_{што}$ -- количесвто делений шкалы точного отсчета -- назначаются из стандартного ряда [10, 20, 50, 100]. $N_{шго}$ -- количество делений шкалы грубого отсчета -- вычислияются не округляется. Обязательно должно выполняться условие:
		$$
		N_{што} > N_{шго}
		$$
		$$
		N_{што} = \frac{N}{N_{шго}} 
		$$
		$$
		H_{што} = H
		$$
		$$
		H_{шго} = H_{што} N_{што} = 2 о. е.
		$$
		Таким образом длина счетной шкалы:
		$$
		L_{што} = N_{што} b = 200мм
		$$
		$$
		L_{шго} = N_{шго} b = 60 мм
		$$
		Тогда диаметры счетных дисков:
		$$
		D_{што} = \frac{L_{што}}{ \pi} \frac{360}{ \psi} = 64 мм
		$$
		$$
		D_{шго} = \frac{L_{шго}}{\pi} \frac{360}{ \psi} = 19 мм
		$$
		Необходимо брать диаметры шкал из ряда [35, 50, 65, 80, 100] мм. В данном случае мы выбираем диаметр 65мм. Тогда нам необходимо пересчитать $b$:
		$$
		b_{што} = \frac{D_{што} \pi \psi_{што}}{360 N_{што}} =  2.04 мм\label{b1}
		$$

		$$\label{b2}
		b_{шго} = \frac{D_{шго} \pi \psi_{шго}}{360 N_{шго}} = 6.8 мм
		$$
		В \ref{b1} мы должны получить число от 1 до 2.5. В то время как \ref{b2} мы не трогаем.
		$$
		i_{што-шго} = \frac{\varphi_{што}}{\varphi_{шго}} = \frac{N_{шго} 360}{ \psi}  = 30
		$$
		
		При разработке механизма отсчетного устройства обычно связывают. Шкалу грубого отсчета с выходным валом, а шкалу точного отсчета с любым предыдущим валом, но при выполнении ряда условий.
		\begin{enumerate}
			\item И ШТО и ШГО должно быть равно расчетной величине
			\item ШГО должна располагаться слева от ШТО со стороны наблюдателя
			\item Шкалы должны вращаться в одну сторону
			\item Шкалы должны быть расположены горизонтально
		\end{enumerate}
		Какие условия нужно будет соблюзти при проектировке
		\begin{enumerate}
			\item И ШТО и ШГО должны быть соот друг другу
			\item ШГО слева от ШТО
			\item ШТО и ШГО должны распологаться на одной оси
		\end{enumerate}
		$$
		x_{max} - x_{min} = \varphi
		$$
	\end{enumerate}
\end{enumerate}

Расчет ЭМП

ЭМП --- (электромеханический привод) --- это устройство состоящее из двух основных частей%тут картинка
\begin{itemize}
	\item Электродвигателя, осуществляющего преобразование электрической энергии в мехнаническую
	\item Редуктор, связывающего электродвигатель с рабочим органом (нагрузкой)
\end{itemize}

Электродвигатели бывают:
\begin{enumerate}
	\item По назначению:
	\begin{enumerate}
		\item Общего назначения. Для работы в нерегулируемом приводе
		\item Исполнительный. Для работы в следящих системах
	\end{enumerate}
	\item По приципу действия:
	\begin{enumerate}
		\item Постоянного тока
		\item Переменного тока:
		\begin{enumerate}
			\item Синхронные
			\item Асинхронные
		\end{enumerate}
		\item Универсальные
		\item Шаговые
	\end{enumerate}
\end{enumerate}

Для того чтобы правильно выбрать электродвигатель необходимо знать:
\begin{enumerate}
	\item Тип привода и режимы его работы (следящий и нерегулируемый привод)
	\item Тип источника питания и его характеристики:
	\begin{itemize}
		\item Для источников постоянного тока необзодимо знать напряжение и ток
		\item Для источников переменного тока (необходимо знать напряжение частоту и ток)
	\end{itemize}
	\item Характеристику нагрузки:
	\begin{enumerate}
		\item Максимальная величина момента
		\item Номинальная углованя скорость вращения
		\item Момент инерции нагрузки
		\item Требуемое быстродействие
	\end{enumerate}
	\item Эксплуатационные условия
	\begin{enumerate}
		\item Окружающая среда (условия работы привода)
		\item Ресурс работы (время жизни в часа)
		\item Массо-габаритные параметры
	\end{enumerate}
\end{enumerate}

Пример: Необходимо выбрать электродвигатель для нерегулируемого нереверсивного привода при следующих исходных данных:
\begin{enumerate}
	\item $M_н = 0.5 н * м$
	\item $J_н = 0.1 кг м^2$
	\item $\varepsilon_н = 10 \frac{рад}{с^2}$
	\item $W_н = 2 \frac{рад}{с} $
	\item $L = 5000 ч$
	\item $t_p = 0.1 с$
	\item $U = 36 В$
\end{enumerate}

Для того чтобы определить двигатель необходимо сначала расчитать требуюмую мощность:
$$
P_н = N_н \omega_н
$$
$$
P_р = \frac{P_н}{\eta} = \frac{N_н \omega_н}{ \eta} = 1.25 Вт
$$

Мы берем кпд равным 0.8


$\xi = (1.2 .. 2.5)$ -- обычная точность
$\xi = (2.5 .. 5)$ -- повышенная точность

$$
1.25 * 1.2 \le P_t \le 1.25 * 2.5
$$
$$
1.5 Вт \le P_t \le 3.125
$$

мы выбрали двигатель Дид-3т по какой-то очень важной причине. Его параметры:
\begin{itemize}
	\item $P_н = 3.6 Вт$
	\item $M_{ном} = 5.6 н * мм$
	\item $n_{ном} = 8000 \frac{об}{мин}$
	\item $M_п = 100 н * мм$
	\item $J_р = 24 * 10^{-8} кг * м^2$
	\item $Т_{эм} = 26 мс$
	\item $U = 36 В$
	\item $Масса = 0.35 кг$
\end{itemize}

Проверка двигателя по моментам. Для нерегулируемого нереврсивного привода характерны:
\begin{itemize}
	\item Продолжительный ресурс работы
	\item Редкие пуски
	\item Отсутсвие реверсов
\end{itemize}
$$
M_п \ge M_\Sigma
$$
$$
M_н \ge M_{стпр}
$$
где:
$$
M_{стрп} = \frac{М_н}{\eta i_0} 
$$
$$
M_{дим пр} = J_{пр} \varepsilon_н i_0 = [(1 + K_м)J_р + \frac{J_н}{i_0^2} ]\varepsilon_н i_0
$$
$$
\omega_{дв} = \frac{n_{ном} \pi}{30}
$$
Таким образом передаточный коэфициент:
$$
i_0 = \frac{ \omega_{дв}}{ \omega_н} = \frac{n_{ном} \pi}{ \omega_н 30} = 420
$$
$$
M_{стрп} = 1.5 Н мм
$$
Приведенный момент инерции считать надо отдельно

$$
J_{пр} = (1 + K_М)J_р + \frac{J_н}{i_0^2} = 104.6 * 10^{-8} кг * м^2
$$

$K_М$ -- коэфициент учитывающий инерционные параметры собственного зубчатого механизма

Подставляем момент динамический приведенный
$$
M_{дин пр} = J_{пр} * \varepsilon_н * i_0 = 4.4 Нмм
$$

Так как условию %надо сослаться на условие
удовлетворяет проверка по моментам пройдена

Для того чтобы узнать время разгона привода вместе с двигателем редуктором и нагрузкой, необходимо найти электромеханическую постоянную привода:

$$
T_{эм} = \frac{J_{пр} \omega_{ном}}{(M_{п} - M_{ном}} = \frac{104.5 * 10^{-8} \pi * 8000)}{(10 -5.6) (10^{-3} * 30} = 0.2 с
$$
Таким образом:
$$
t_p = 3 * T_{эм} = 3 * 0.2 = 0.6 c
$$
Что не удовлетворяет условиям задачи. Следовательно по вермени разгона двигатель подобран неверно

Пример №2:

Выбрать двигатель для следящего регулируемого привода, при следующих исходных данных:
\begin{itemize}
	\item $M_н = 1 \frac{н}{м} $
	\item $\omega_н = 4 \frac{рад}{с}$
	\item $J_н = 0.25 кг * м^2$
	\item $\varepsilon_н = 10 \frac{рад}{с^2} $
	\item $t_p = 0.5 c$
	\item $L = 500 ч$
	\item $U = 27 В$
\end{itemize}
Предварительно выбираем серию ДПР

Проводим предварительные расчеты:
\begin{enumerate}
	\item Расчет мощности
	$$
	P_р = \frac{М_н \omega_н}{\eta_0} = 5 Вт 
	$$
	$$
	6 Вт \le P_t \le 12.5 Вт
	$$
\end{enumerate}
Исходя из предварительного расчета выбираем двигатель ДПР62-02

Его паспортные данные:
\begin{itemize}
	\item $n_{ном} = 6000 \frac{об}{мин} $
	\item $P_н = 12.3 Вт$
	\item $M_{ном} = 19.6 Н мм$
	\item $J_p = 0.036 * 10^{-4}$
	\item $M_п = 137 Н * мм$
	\item $L = 1000 ч$
\end{itemize}
**здесь н это номинальный, а в условиях задачи н -- нагрузка

Для следящего привода характерны:
\begin{enumerate}
	\item постоянно-кратковременные режимы работы
	\item высокое быстродействие
	\item большая частота пусков и реверсов.
\end{enumerate}

Формула проверки момента номинального для следящего привода:
$$
M_{ном} \ge M_{\Sigma_{пр}} = M_{ст.пр} + M_{дин. пр.}
$$
передаточное отношение:
$$
i_0 = \frac{n_{дв}}{n_н} = 150
$$
моменты приведенные:
\begin{itemize}
	\item $M_{ст.пр.} = \frac{M_{ном}}{\eta_0 i_0} = 8.3 Н мм$
	\item $M_{дин.пр.} = J_{пр} \varepsilon_н i_0 = 25.6 Н мм$
	\item $J_{пр} = (1 + K_м) J_p + \frac{J_н}{i_0^2}$
\end{itemize}
$$
19.6 \ge 34.1
$$
Проверку по моментам не прошел, выбираем другой двигатель

\begin{enumerate}
	\item Отсчетное устройство
	\item двигатель
	\item Кинематический расчет (раскладываем $i_0$ по ступеням) $n$ -- Количество ступеней, которые нам необходимо посчитать
	Для расчета $n$ можно использовать:
	\begin{itemize}
		\item Формулы
		\item Монограммы (см. в конце методы)
		\item "на пальцах"
	\end{itemize}
	$$
	i_{ост} = \frac{i_0}{i{што-шго}} 
	$$
	$$
	n_{ост} = 1.85 \log_{10}{ i_{ост}}
	$$
	Необходимо так же посчитать:
	\begin{enumerate}
		\item Моменты
	\end{enumerate}
	\item Расчет модуля $m$
	\item Геометрические расчеты
	\item Расчет валов $d \ge 4 мм$
	\item Расчет опор
	\item Расчет на точностьы
	\item Эскиз
\end{enumerate}
Расчет электромеханического привода на точность.

Основная задача расчета --- сводится к проверке выполнения следующего неравенства:
$$
\Delta_{\Sigma} \le [ \delta_0 s]
$$

Основными прогрешностями зубчатой передачи являются погрешности возникающи из-за мертвого хода $ \Delta_{\Sigma} = \Delta_{мх} + \varphi_{ior}$ Погрешность мертвого ( $ \Delta_{мх}$) хода делится на: люфтовую погрешность и погрешность упрогого мертвого хода. $\varphi_{ior}$ -- погрешность кинематическая.

План действий:
\begin{enumerate}
	\item Необходимо выбрать вид сопряжения и степень точности:
	Виды сопряжения:
	\begin{itemize}
		\item H -- $J_n = 0$
		\item G
		\item F
		\item E
		\item D
	\end{itemize}
	От H до D $j_n$ растет.

	Используемые нами степени точности:
	$6 \to 7 \to 8 \to 9$
	Восьмую степень точности можно получить просто режущем инструментом. Седьмая степень точности требует шлифовки. Шестая степень точности требует полировки.

	Исходя из влияния высших сил выберем 7F.
	\item Определение кинематической погрешности:

	\underline{Кинематическая погрешность передачи} представляет собой разность между действительным и номинальным (расчетным) углами поворота \underline{ведомого} колеса. Кинематическая погрешность зубчатых колес расположенных на одном валу суммируется. А общая погрешность находится как сумма всех погрешностей приведенных к одному, обычно, выходному валу.
	% Здесь должен быть рисунок
	Нумеруем валы от двигателя римскими цифрами.
	По некоторому условию:
	Зубчатые колеса:
	\begin{itemize}
		\item $z_1 = 26$
		\item $z_2 = 95$
		\item $z_3 = 26$
		\item $z_4 = 132$
		\item $z_5 = 25$
		\item $z_6 = 175$
	\end{itemize}
	Передаточный коэфиент
	\begin{itemize}
		\item $i_{12} = 3.6$
		\item $i_{34} = 5.1$
		\item $i_{56} = 7$
	\end{itemize}
	
	Модуль -- $m = 0.4$
	$$
	i_0 = i_{12} i_{34} i_{56} = 130
	$$
	$$
	\varphi_6 = 270
	$$
	\item Определим делительные диаметры зубчатых колес:
	$$
	d_1 = 0.4 *25 = 10,4 мм
	$$
	$$
	d_3 = 9.4 * 26 - 10.4 мм
	$$
	$$
	d_5 = 0.4 * 25 = 10 мм
	$$
	$$
	d_2 = 0.4 * 95 = 39 мм
	$$
	$$
	d_4 = 0.4 * 132 = 52.8 мм
	$$
	$$
	d_6 = 0.4 * 175 = 70 мм
	$$
	Далее 2 пути расчета:
	\begin{itemize}
		\item Метод максимума минимума

		Допуск на кинематическую погрешность колеса находят как сумму допусков на накопленную погрешность шага -- $F_p$ и допуска на погрешность профиля зуба -- $f_f$
		$$
		F_i' = F_p + f_f
		$$
		Согласно таблице П2.1 из методы кокорева:
		$$
		F_1' = F_3' = F_5' = 22 + 9 = 31 мкм
		$$
		$$
		F_2' = 30 + 9 = 39 мкм
		$$
		$$
		f_4' = 35 + 9 = 45 мкм
		$$
		$$
		f_6' = 35 + 9 - 45 мкм
		$$

		Допуск на уголовую кинематическую погрешность в угловых минутах находят как:
		$$
		\Delta \varphi_i = \frac{6.88 F_i'}{m z}
		$$
		Тогда допуск на кинематическую погрешность для каждой из шестерней:
		$$
		\Delta \varphi_1 = 20.5'
		$$
		$$
		\Delta \varphi_3 = 20.5'
		$$
		$$
		\Delta \varphi_5 = 21.3'
		$$
		$$
		\Delta \varphi_2 = 7.06'
		$$
		$$
		\Delta \varphi_4 = 5.7'
		$$
		$$
		\Delta \varphi_6 = 4.3'
		$$
		
		Суммарная кинематическая погрешность:
		$$
		\Delta \varphi_{io\Sigma} = \sum\limits_{j=1}^{N} \frac{ \Delta \varphi_{ij}}{i_{j-N}} K_{\varphi j}
		$$

		Где $K_{\varphi}$ -- коэфициент учитывающий зависимость кинематической погрешности расчитываемой передачи от фактического максимального угла поворота колеса. Определяется по таблице 2.25 с. 21

		$$
		i_{1-4} = i_0 = 130
		$$
		$$
		i_{2-4} = \frac{i_0}{i_{12}} = 36
		$$
		$$
		i_{3-4} = \frac{i_0}{i_{12} i_{34}} = i_{56} = 7
		$$
		$$
		i_{4-4} = 1
		$$

		Таким образом формула %ссылка на формулу
		имеет вид:
		$$
		\Delta \varphi_{io\Sigma} = \frac{ \Delta \varphi_1}{i_{1-4}} K_{\varphi_1} + \frac{ \Delta\varphi_2 + \Delta\varphi_3}{i_{2-4}} K_{\varphi_2} + \frac{ \Delta\varphi_4 + \Delta\varphi_5}{i_{3-4}} K_{\varphi_3} + \frac{ \Delta\varphi_4}{i_{4-4}} K_{\varphi_4} = 8.45'
		$$
		Где:
		$$
		K_{\varphi_4} = 0.85
		$$
		Поскольку таблица расчитана до 360 градусов мы принимаем $K_{\varphi}$ для углов, мера которых больше 360, равным единице! Тогда:
		$$
		K_{\varphi_3} = K_{\varphi_2} = K_{\varphi_1} = 1
		$$
		$ \Delta \varphi_{io\Sigma} = 8.45'$ --- кинематическая погрешность методом максимума-минимума
		\item Вероятностный метод.
		$$
		\Delta\varphi_{max i} = \frac{6.88 K F_i'}{m z_i} 
		$$
		Коэфициент $K$ -- коэфициент фазовой компенсации
		$$
		\Delta\varphi_{1 max} = 19.7'
		$$
		$$
		\Delta\varphi_{max 3} = 20.1'
		$$
		$$
		\Delta\varphi_{max 5} = 20.9'
		$$
		$$
		\Delta\varphi_{max 2} = 6.8'
		$$
		$$
		\Delta\varphi_{max 4} = 5.6'
		$$
		$$
		\Delta\varphi_{max 6} = 4.2'
		$$
		Теперь считаем минимальное значение:
		$$
		\Delta\varphi_{i min} = \frac{4.88 K_s F_i'}{m z_i}
		$$ --- для 7, 8 классов точности
		$$
		\Delta\varphi_{i min} = \frac{4.3 K_s F_i'}{m z_i} 
		$$ --- для 5, 6, 9, 10 классов точности
		Поскольку у нас 7 класс точности используем формулу %Ссылка на формулу
		$$
		\Delta\varphi_{1 min} = 11.6'
		$$
		$$
		\Delta\varphi_{2 min} = 4'
		$$
		$$
		\Delta\varphi_{3 min} = 12.4'
		$$
		$$
		\Delta\varphi_{4 min} = 3.46'
		$$
		$$
		\Delta\varphi_{5 min} = 15.0'
		$$
		$$
		\Delta\varphi_{6 min} = 3'
		$$
		Найдем поле рассеивания:
		$$
		V_1 = 19.7 - 11.6 = 8.1'
		$$
		$$
		V_3 = 20.1 - 12.4 = 7.7'
		$$
		$$
		V_5 = 20.9 - 15 = 5.9'
		$$
		$$
		V_2 = 6.8 - 4 = 2.8'
		$$
		$$
		V_4 = 5.6 - 3.5 = 2.1'
		$$
		$$
		V_6 = 4.2 - 3 = 1.2'
		$$
		Координаты середины поля рассеивания:
		$$
		E_i = \frac{ \Delta\varphi_{max i} + \Delta\varphi_{min i}}{2}
		$$
		$$
		E_1 = 15.7'
		$$
		$$
		E_3 = 16.3'
		$$
		$$
		E_5 = 18'
		$$
		$$
		E_2 = 5.4'
		$$
		$$
		E_4 = 4.6'
		$$
		$$
		E_6 = 3.6'
		$$
		Суммарная вероятностная кинематическая погрешность передачи:
		$$
		\Delta\varphi_{i \Sigma}^B = \sum\limits_{j = 1}^{N} \frac{E_{ij}}{i_{j-N}} + t_1 \sqrt{ \sum\limits_{j = 1}^{N} \left( \frac{V_{ij}}{i_{j-N}} \right) } = 7.96
		$$
		$t$ --- коэфициент стьюдента. В домашнем задании вероятность принимаем равной $p = 10\%$

	\end{itemize}
	\item Определение погрешностей вносимых мертвым ходом
	Теоретически угол поворота ведомого колеса связа с углом поворта ведущего колеса линейной зависимостью. Практически при попроте ведущего колеса на некоторый угол $ \delta\varphi_1$ ведомое колесо может оставаться неподвижным из-за наличия бокового зазора -- люфта между зубьями сопряженных колес, а так же из-за наличия упругих деформаций рабочих участков валов, передающих крутящий момент. Величина $ \delta\varphi_1$ является погрешностью мертвого хода. Мертвый ход равен сумме двух погрешностей -- люфтовой погрешности и погрешности упругого мертвого хода.

	Определим межосевые расстояния:
	$$
	a_{12} = m * 0.5 (z_1 + z_2) = 24.2 мм
	$$
	$$
	a_{34} = m * 0.5 (z_3 + z_4) = 31.6 мм
	$$
	$$
	a_{56} = m * 0.5 (z_5 + z_6) = 40 мм
	$$

	Собственные люфтовые погрешности передачь отнесенные к ведущим колесам (шестерням) каждой пары:
	$$
	\Delta\varphi_л = \frac{7.33 j_{n max}}{m z_1}
	$$
	Где $z_1$ -- ведущее колесо в паре
	$j_{n max}$ -- максимальный боковой зазор
	$$
	j_{n max} = 0.7 (E_{HS1} + E_{HS2}) + \sqrt {0.5 (T_{H_1}^2 + T_{H_2}^2) + 2 (f_a)^2}	
	$$
\end{enumerate}

Есть некоторые данные:
\begin{itemize}
	\item $i_0 =$
	\item $i_{што-шго}$
\end{itemize}
тогда нам необходимо сделать следующие шаги:

\begin{enumerate}
	\item $i_{ост} = \frac{i_0}{i_{што-шго}} = сколько-то \to n = 1.85 \log_{10} i_{ост} = что-то$
	Если получается надо брать передаточные отношения так, чтобы они возрастали от двигателя
	$n$ -- количество ступеней
	$$
	i_j = \sqrt[n] {i_{ост}}
	$$
	\item Подбираем колеса $z_1 = z_3 = z_5 = 20$
	Тогда:
	$$
	z_2 = i_{12} z_1 = чему-то
	$$
	$$
	z_4 = i_{34} z_3 = чему-то
	$$
	$$
	z_6 = i_{56} z_5 = чему-то
	$$
	Далее считаем следующую величину (практический коэфициент передачи ?)
	$$
	i_{пр} = \frac{z_2}{z_1} \frac{z_4}{z_3} \frac{z_6}{z_5} = чему-то
	$$
	$$
	\xi = \frac{i_0 - i_{пр}}{i_0} 100 \% = чему-то
	$$
	\item Рассчитать моменты на каждом из зубчатых колес:
	$$
	M_н = 1 н * м = M_8
	$$
	$$
	M_7 = \frac{M_8}{i_{78} \eta_{пер} \eta_{опор}} 
	$$
	$$
	M_6 = M_7
	$$
	$$
	M_5 = \frac{M_6}{i_{56} \eta_{пер} \eta_{опор}} 
	$$
	И так далее, ввесело что капец!
	\item Считаем модули:
	$$
	m_1 =
	$$
	$$
	m_3 =
	$$
	$$
	m_5 =
	$$
	$$
	m_7 = 
	$$
	Ну мы их посчитали и они какие-то получились
	$$
	m = K_м \sqrt[3] { \frac{M Y_F K}{z \psi_m [ \sigma_F]}} 
	$$
	$$
	\psi_m = \frac{b}{m} = 8
	$$
	\item Расчет на контактную прочность
	\item Геометрические параметры:
	$d$, $d_f$, $d_a$, $d_b$, $h$, $h_a$, $h_f$, $b$, $a$, $m_{теор}$
	\item Расчет на точность
	\item Примерный диаметр вала $d$
	\item Расчет опор
	\item Расчет потенциометра
	\item Миллиметровка
	\item к 8 пункту ( $ \Delta \varphi_{умх}$) Вала на прочность (эпюры) Опоры (расчет)
\end{enumerate}

Упругие чувствительные элементы.

Упругие чувствительные элементы служат для преобразования измеренного давления или силы в какие-либо механические перемещаения. К ним относятся: пружины (прямые, плоские, изогнутые, спиральные, винтовые); мембраны и мембранные коробки; гофрированные трубки; трубчаные сифоны
Характеристика упругого чувствительного элемента --- зависимость между его прогибом (ходом) и вызывающей этот прогиб нагрузкой.

Пружины делятся на:
\begin{itemize}
	\item По форме:
	\begin{itemize}
		\item Прямые
		\item Изогнутые
		\item Спиральные
		\item Винтовые
	\end{itemize}
	\item По назначению:
	\begin{itemize}
		\item Силовые (аккумуляторы энергии)
		\item Измерительные (упругие чувствительные эелементы)
		\item Пружины для соединения деталепй
	\end{itemize}
\end{itemize}

Расчет винтовых пружин растяжения сжатия.

Параметры:
\begin{itemize}
	\item $D$ -- диаметр пружины
	\item $d$ -- диаметр проволоки
	\item $ \alpha$ -- угол наклона витка
\end{itemize}

\begin{enumerate}
	\item $P_{max} = \frac{ \pi d^3}{8 D_0} [\tau]_{кр}$
	\item $\tau_{max} = k \frac{8 P_{max} D_0}{\pi d^3} \le [\tau]_{кр} $ 
	$K$ -- коэфициент зависящий от индекса пружины.
	\item $k = \frac{4 c - 1}{4 c - 4} + \frac{0.615}{c} $, где $c = \frac{D_0}{d} $
	\item Диаметр проволоки:
	$$
	d \ge \sqrt { \frac{8 D_{max} c k}{n [\tau_{кр}]} }
	$$
	Осевое перемещение:
	$$
	f = \frac{8 D_0^3 P_{max} n}{G d^4} = \frac{8 c^3 p_{max} n}{G d} 
	$$
	\item Жесткость
	$$
	K = \frac{G d^4}{8 n D_0^3}
	$$
	\item Длина развернутой проволоки
	$$
	L = \frac{\pi D_0 n_0}{\cos{\alpha}} 
	$$
\end{enumerate}

Задача: Определить размеры винтовой пружины сжатияпри установке ее в прибор.
\begin{itemize}
	\item $P_{min} = 24 H$
	\item $P_{max} = 26 H$
	\item $f_{раб} = 0.5 мм = 0.5 * 10^{-3} м$
	\item $ОВС$ - марка стали
	\item $G = 7.8 * 10^{10}$
	\item $ \sigma_{в} = 188 * 10^7 Па$
	\item $ \alpha = 7$
\end{itemize}

%рисунок
$$
\frac{f_{min}}{f_{раб}} = \frac{f_{min}}{P_{max} - P_{min}}
$$
Следовательно:
$$
f_{min} =  \frac{P_{min} P_{раб}}{P_{max} - P_{min}} = 6 мм
$$
$$
f_{max} = f_{min} + f_{раб} = 6.5 мм
$$
$$
d \ge \sqrt { \frac{8 P_{max} C K}{\pi [\tau]} }
$$
Критическое напряжение:
$$
[\tau_{пр}] = \frac{\sigma_B}{n_т} = 47 * 10^7 \frac{н}{м^2} 
$$
Индекс пружины:
$$
c = 8 = \frac{D_0}{d} 
$$
Коэфициент зависящий от индекса пружины
$$
k = \frac{4c - 1}{4c - 4}  + \frac{0.615}{c} = 1.18
$$
$$
d \ge \sqrt { \frac{8 * 26 * 1.18 8}{3.14 * 47 * 10^7}} = 1.15 * 10^{-3} = 1.15 мм
$$
Согласно какому-то там госту выбираем прволоку $d = 1.2 мм$
$$
D_0 = d с = 1.2 * 8 = 9.6 мм
$$
Определяем количество витков
$$
n = \frac{1}{8} \frac{G f_{max} d }{P_{max} C^3} = \frac{1}{8} \frac{7.8 * 10^{10} * 1.2 * 10^{03} * 6.5}{26 * 8^3} = 5.8 = 6
$$
С учетом наличия опорной поверхности:
$$
n_0 = n + 2 = 6 + 2 = 8
$$
Длина заготовки:
$$
L = \frac{\pi D_0 n_0}{\cos{ \alpha}} = \frac{8 * 9.6 3.14}{cos{7}} = 243 мм
$$

Выбор и расчет люфтовыбирающих колес

Выбор бокового зазора в зубчатом зацеплении с использованием пружины растяжения.
%картинка

Две составные части колеса (1) (2) скреплены спенциальными винтам (4), предотвращающими осевое перемежение одной детали по отношению к другой и допускающие угловой поворот одной части по отношению ко второй. Пружина (3) помещены в специальные окна   стягивают половины колеса зацепами, зацепляемыми за отверстия в усиках половин. Предварительно части (1) и (2) смещают отностилеьно друг друга в направлении растяжения пружины (3) на несколько угловых шагов $p$. И в таком положении их вводят в зацеплении с ведущим колесом. Зацепление обеспечивается моментом развиваемым пружинами на плече $2A$. Величина этого момента должна превышать величину передаваемого моента в $1.5-2$ раза. 

% Картинка с пружиной
$$
OB = OE = OK = R
$$
$$
z = 80
$$
$$
m = 1 mm
$$
$$
M_{кр} = 150 н мм
$$
При взаимном повороте взаимных частей (1) и (2) начальная длина пружины равная $BE$ увеличится до $L_1$ равное $BK$ следовательно в пружинах возникнет стягивающее усилие. При этом первоначально действовавашее плечо $A$ равное $OM$ уменьшится до $A_1 = ON$
$$
A_1 = R \cos{ \frac{ \alpha \phi}{2} } = R (\cos{ \frac{ \alpha}{2}}\cos{ \frac{ \phi}{2} } - \sin{ \frac{ \alpha}{2} } \sin{ \frac{ \phi}{2} })
$$
$$
\cos{ \frac{ \alpha}{2} } = \frac{A}{2}; \sin{ \frac{ \alpha}{2} } = \frac{L}{2R} 
$$
$$
A_1 = A \cos{ \frac{ \phi}{2}} - \frac{L}{2} \sin{ \frac{ \phi}{2} }
$$
Угол $ \phi$ это угол взаимного смещения колес (1) и (2). Пара сил двух пружин действующих на плече $2A_1$ создают зазоровыбирающий момент $M^{'} = 2 A_1 P_2$ следовательно $P_2 = \frac{M^{'}}{2A_1} $
$$
M^{'} = (1.5 .. 2) M_{кр} = 2 * M_{кр} = 2 * 150 = 300 Н * мм
$$
Отслюда:
$$
P_2 = \frac{2 M_{кр}}{2 A \cos{ \frac {\phi}{2}} - L \sin{ \frac{ \phi}{2} }} 
$$
Т к взаимное смещение колес (1) и (2) должно быть проведено на целое количество зубьев, то величина угла $ \phi$ можно выразить следующим образом:
$$
\phi = \frac{360 p n}{ \pi D } = \frac{m 360 \pi n}{ \pi D } = \frac{360 m n}{m z} = \frac{360 n}{z}
$$
$$
P_2 = \frac{2 M_{кр}}{2A\cos{ \frac{180 n}{2} } - L \sin { \frac{180 n}{z}} } 
$$
Необходимо задаться контрукционными параметрами: Длина пружины $L$, плечо Ф и количество зубьев на которые мы раздвигаем n. Необходимо ориентироваться на короткие и жесткие пружины с индексом пружины $C \le 10$. Пусть $n = 4$, $A = 30$, $L = 12$
$$
P_2 = \frac{2 * 150}{2 * 30 * \cos {9} - 12 * \sin {9}} = 5.23
$$
Необходимо теперь посчитать пределную силу:
$$
P_3 = \frac{P_2 }{1 - \delta} = \frac{5.23}{1 - 0.1} = 5.8
$$
Открываем гост 13771 там выбираем максимальную силу:
$$
P_3 = 6
$$
Силой небес выбираем пружину №92
Ее параметры:
\begin{itemize}
	\item $P_3 = 6$
	\item $d = 0.4$
	\item $D = 4.5$
	\item $K_1 = 3.71$
	\item $f_1 = 1.648$
\end{itemize}

Считаем рабочую деформацию
$$
f_{раб} = L_1 - L = L \cos{9} + 2A \sin{9} - L = 9.2 мм
$$
Жескость всей пружины:
$$
K = \frac{P_2}{f_{раб}} = 0.57
$$
Число витков рабочих:
$$
n = \frac{K_1}{K} = \frac{3.71}{0.57} = 6
$$
Добавляем $2.5$ витка так, как концы пружин в разные стороны:
$$
n_0 = n + 2.5 = 6.5
$$
% Картинка
$$
H_0 = (n + 1) d = (6 + 1) 0.4 = 2.8
$$
$$
L_0 = H_0 + 2 D = 2.8 + 2 * 4.5 = 11.8
$$

Разработка чертежа общего ввида
Для разработки чертежа общего вида, первым делом надо проанализировать кинематическую схему
% Картинка
1 - двигатель

Устройство управления должно содержать в себе прибор, вырабатывающий эелктрический сигнал зависящий от угла поворота вала на котором он установлен.

Подбор потенциометра.

Основные параметры потенциометра:
\begin{enumerate}
	\item Угол поворота вала потенциометра
	\item Разрешающая способность (точность) потенциометра
\end{enumerate}
По углу поворота потенциометры бываюте:
\begin{enumerate}
	\item Меньше 1 поворота
	\item Однооборотные потенциометры
	\item Пятиоборотные потенциометры
	\item Десятиоборотные потенциометры
	\item Двадцатиоборотные потенциометры
\end{enumerate}

2 вида:
\begin{enumerate}
	\item ППМЛ - 5-10 оборотов
	\item ПТП - однооборотные
\end{enumerate}

Выбот потенциометра по углу поворота:
\begin{enumerate}
	\item $0.8 \phi_{пм} \le \phi_{в} \le 1 \phi_{пм}$
\end{enumerate}

\begin{enumerate}
	\item Главный вид
	\item Равертка (сложный ступенчатый разрез по осям валов)
	\item Дополнительные виды для понимания расположения элементов
	\item Таблица составных частей
	\item Схема деления на составные части
\end{enumerate}

Таблица составных частей должна состоять из графов, такие как:
\begin{itemize}
	\item Заимствованные изделия
	\item ВНовь разработанные изделия
	\item Стандартные изделия
	\item Прочие изделия
\end{itemize}

Единственным заимствованным изделием являются стойки.

Схема деления отображает структуру прибора. На схеме делений можно отобразить уровневость сборки.

Разработка чертежа общего вида начинается с прорисовки главного вида (вид сверху вероятно)

Муфты.

Муфты бывают:
\begin{enumerate}
	\item Соединительные
	\item Предохранительные:
	\begin{enumerate}
		\item Автоматические
		\item Муфты с разрушаемым элементом
	\end{enumerate}
\end{enumerate}

Бывают 3 типа погрешностей при соединении валов:
\begin{enumerate}
	\item $ \Delta x$ -- экстринситет -- несоостность
	\item $ \Delta z$ -- расстояние между валами
	\item $ \Delta \alpha$ -- угол между осями
\end{enumerate}

Муфты соединительные:
\begin{enumerate}
	\item Втулочная: На 2 вала надевается втулка. Закрепление втулки на валах выполняется штифтованием. Причем для снижения излома штифтов они монтируются в перпендикулярных плоскостях. В металлическом исполнении позволяет решить погрешность $ \Delta z$
	\item Поводковая или пальцевая муфта, На половине муфты есть палец и луза, и две половины соединяются пальцем к лузе, тем самым позволяя решить $ \Delta x$ $ \Delta z$ $ \Delta \alpha$.
	\item Поводковая муфта с эластичным элементом: Между 2мя полумуфтами расположен эластичный материал и отсутствуют лузы, есть только штифты, таким образом позволяет нивилировать импульсы от резких пусков и стопов.
\end{enumerate}

Предохранительная муфта с разрушаемым элементом
Для дз:
$$
D_{штифта} = \frac{1}{3} ... \frac{1}{4} d_{вала}
$$

Конструкционно выглядит как Втулочная муфта, только 2й штифт выполняется с расчетом на то что его срежет при определенном моменте достигнутом в процессе эксплуатации.

Муфта предохранительная пружинная

Предохраняет от кратковременных перепадов нагружения.

Сложные передачи.

Бывают:
\begin{itemize}
	\item Сложные рядные передачи;
	\item Сложные многоступенчатые передачи;
	\item Сложные планетарные передачи;
	\item Сложные волновые передачи;
	\item Дифференциальные передачи.
\end{itemize}

Рядные передачи.

Это соединенные в ряд зубчатые колеса
Передаточное отношение:
$$
i_{1n} = (-1)^k \frac{z_{n}}{z_1} 
$$
Где $n$ -- число зубчатых колес, $k$ -- число передачь.
Колеса между входным и выходным колесами я вляются паразитными.

Кпд передачи:
$$
\eta = \eta_{12} \eta_{23} \eta_{34} \eta_{45} \eta_{56} \eta_{67}
$$

Применяют в трех случаях:
\begin{enumerate}
	\item Позволяют вписывать передачу в заданные межосевые расстояния;
	\item Когда необходимо согласовать вращения входного и выходного вала;
	\item Служат для обхода препятствий внутри конструкции.
\end{enumerate}

Достоинства и недостатки:
\begin{itemize}
	\item Возможность согласования валов на определенном межосевом расстоянии;
	\item Возможность смены направления вращеия передачи без пересчета передаточного отношений;
	\item Возможность обхождения препятствий внутри конструкции;
	\item Возможность снятия нагрузки с нескольких выходных валов при одном входном валу;
	\item В передаточном отношении участвуют только первое и последнее зубчатое колесо, все остальные колеса являются паразитными;
	\item Низкое кпд;
	\item Большое количество промежуточных элементов.
\end{itemize}

Многоступенчатые передачи.

Сферы применения:
Повсеместно, где необхобходимо получить большое передаточное отношение и кпд.

\begin{enumerate}
	\item Возможность снятия нагрузки с нескольких выходных валов при одном входном.
	\item Возможность получить большое передаточное отношения.
	\item Возможность получения редуктора с большим передаточным отношением и относительно небольшими габаритами.
	\item Простота расчета, простота сборки.
	\item Резкий спад кпд при большом передаточном отношении.
\end{enumerate}

Планетарные передачи.
% Необходимо посмотреть в тытрубе как работает планетарная передача и дифференциальная передачи и волновая передача

Планетарным называется механизм состоянищий из зубчатых колес и вращающихся звеньев, на которых располагаются оси зубчатых колес.
Звено на котором располагаются подвижные оси колес называется водило, обозначается $H$. А зубчатые колеса с подвижными осями называются планетами или сателитами.
Колеса с неподвижной осью вращения называется центральным или солнечным. Неподвижное колесо называется опорным или корона.
\end{document}
