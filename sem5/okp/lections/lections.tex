\documentclass{article}
\usepackage[utf8]{inputenc}
\usepackage[english,russian]{babel}
\begin{document}
%Тут должен быть рисунок
Линия зацепления $n_1n_2$ --- тракетория общей точки контакта $K$ зубьев при вращении колес при передаче.
$n_1n_2$ --- теоретическая линия зацепления, ее обозначают $g$
В реальности контакт между зубьями будет на отрезке $AB$ называемый активной линией зацепления.
$\varphi_{\gamma}$ - угол перекрытия зубчатых колес.
Угол перекрытия показывает на каком углу колесо входит в зацепление и выходит из него.
Для более плавной и качественной передачи момента коэфициент перекрытия должен быть больше $1$, обычно $1.2$.
$\alpha$ --- угол зацепления (угол между коризонтальной прямой и теоритической линией зацепления).
$AB = g\alpha$
$\varepsilon_{\gamma}=\frac{\varphi_{\gamma}}{\tao}$
Межосевое расстояние $a$ это расстоянием между осями зубчатых колес. наиболее часто применяют зубчатые колеса с так называемым делительным межосевым расстоянием. При этом в полюсе контакта касательными являются делительные окружности.
В данном случае они же и являются начальными (центроидами)
$a = r_1 + r_2 = \frac{mz_1}{2} + \frac{mz_2}{2}$
Анализ сил и моментов в одноступенчатой зубчатой передаче.
%Здесь должна быть картинка
$F_{n_1} = F_{n_2}$
$F_{n_1} = \frac{2m_1}{d_1}$
$F_{n_2} = \frac{2m_2}{d_2} \to \frac{m_2}{m_1} = \frac{d_2}{d_1}=i_{12}$
Расчет модуля зубчатых колес
%Здесь должна быть еще одна картинка
$F_n \cos{\alpha}$ - полезная сила
$F_n \sin{\alpha}$ - бесполезная сила  
Так как материалы при сжатии выдерживают нагрузку намного больше чем при растяжении, под опасной точкой будем понимать точку $A$.
$F_n = \frac{2M_2}{mz_2\cos{\alpha}} = \frac{2M_1}{mz_1\cos{\alpha}}$
$\sigma_u = $
$\sigma_p = \frac{\sigma_{пред}}{n_т}$ 
$\sigma_{пред} \to \sigma_т,\sigma_В,\sigma_{пп}$
$n_т = 1.5$
$\sigma_\Sigma = \sigma_{cж} - \sigma_{и}$
$\sigma_{сж} = \frac{f\sin{\alpha}}{bs}$
$b = \Psi_{вм}m$
$\Psi_{вм} = \frac{b}{m}$
$\Psi_{вм} = 3..16$
$W_{изг} = \frac{bs^2}{6}$
$M_{изг} = h_рF_n\cos{\alpha}$
$\sigma_и = \frac{6h_рF_n\cos{\alpha}}{bs^2}$
$\sigma_р = \frac{F_n\sin{\alpha}}{bs}$
$\sigma_{\Sigma} = \frac{6h_рF_n\cos{\alpha}}{bs^2} - \frac{F_n\sin{\alpha}}{bs} = \frac{F_n}{b}(\frac{6h_р\cos{\alpha}}{s^2} - \frac{F_n\sin{\alpha}}{s})$
$F_n = \frac{2M_2K}{mz_2\cos{\alpha}}$
$K = K_bK_V$
$K_b$ - коэфициент концентричности напряжения
$K_V$ - кожфициент динамичности нагрузки
%Еще много всяких формул (жуть)
Основной формулой для расчета контактных напряжений является формула Герца для контакта 2х целинтров:
$\sigma_M=sqrt{2}{\frac{E_прq_n}{\rho_пр(1-\mu^2)2\pi}}$
$E_{пр} = \frac{2E_1E_2}{E_1 + E_2}$
$\rho_{пр} = \frac{\rho_1\rho_2}{\rho_1 + \rho_2} = \frac{1 + i_{12}}{r_1i_{12}}$%дописать
$q_n = \frac{F_n}{b}$
$d_2 = 2r_2 = 2i_{12}r_1 = \frac {2 a i_{12}}{i_{12} + 1}$
$q_n = \frac {2M_2 K}{d_2 b \cos{\alpha}} = \frac{2 M_2 k (i_{12} + 1)}{b \cos{\alpha} 2 a i_{12}}$
$\sigma_M = \sqrt{2}{\frac{2 E_1 E_2 2 M_2 K (i_{12} + 1) (1 + i_{12})}{(E_1 + E_2) b \cos{\alpha} 2 a i_{12} a i_{12} \sin{\alpha}2 \pi (1 - \mu^2)}}$
Приводя некоторые сокращения:
$\sigmaM = \sqrt{2}{\frac{2 E_1 E_2 2 M_2 K (1 + i_{12})^3}{(E_1 + E_2) \psi_{ba} \sin {2\alpha} a^3 i_{12}^2 2 \pi (1 - \mu^2)}}$
$z_M = \sqrt{2}{\frac{2}{\sin{2\alpha}}}$
$z_m = \sqrt{2}{\frac{E_{пр}}{2\pi(1-\mu^2)}}$
Таким образом:
$\sigma_M = z_M z_m z_\varepsilon - \sqrt{2}{\frac{k M_2 ( 1 _ i_{12})^3}{\Psi_ba a^3 i_{12}^2}}$
Тогда:
$a = (i_{12} + 1)\sqrt{3}{\frac{K M_2}{\Psi_ba}(\frack{z_m z_M z_\varepsilon}{i_{12}[\sigma_M]})^2}$
Если рассечь косозубое колесо перпендикулярно оси зубьев, то в сечении будет эллипс и эвольвентный профиль зубьев
\end{document}
