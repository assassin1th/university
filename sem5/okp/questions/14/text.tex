\documentclass{article}
\usepackage{amsmath}
\usepackage{mathtext}
\usepackage[english,russian]{babel}
\usepackage[T2A]{fontenc}
\usepackage[utf8]{inputenc}
\begin{document}
\section{Расчет на точность. Метод минимум-максимума}

Допуск на кинематическую погршность колеса находят как сумму допусков на накопленную погрешность шага $F_p$
и допуска на погрешность профиля зуба $f_f$
$$
F_{f}^{'} = F_{i\:i} + F_{f\:i}
$$
Допуск на угловую кинематическую погрешность в угловых минутах находят как:
$$
\Delta \varphi_i = \frac{6.88 \cdot F_i^{'}}{m z_i} 
$$
Суммарная кинематическая погрешность:
$$
\Delta \varphi_{i\:o\:\Sigma} = \sum\limits_{j = 1}^{N} \frac{ \Delta \varphi_{i\:j}}{m z_i}
$$
$K_\varphi$ -- коэффициент, учитывающий зависимость кинематической погрешнсоти рассчитываемой передачи от максмального 
угла поворота колеса.

\underline{Опр-е погрешностей вносимых мертвым ходом}

Собственный люфтовые погрешнсоти передачи отнесенные к ведущим колесам (шестерням) каждой пары:
$$
\Delta \varphi_{л \: i} = \frac{7.33 \cdot j_{n\:max\:i}}{m z_i} 
$$
$j_{n\:max}$ -- максимальный боковой зазор
$$
j_{n\:max} = 0.7 \left (E_{HS_1} + E_{HS_2}\right) + \sqrt[2]{0.5\cdot\left(T_{H_1}^2 + T_{H_2}^2\right) + 2 * \left(f_a\right)^2}
$$
$E_{HS_1}$, $E_{HS_2}$ -- наименьшее смещение исходного контура шестерни и колеса
$T_{H_1}, T_{H_2}$ -- допуск на смещение исх. контур шестерни и колеса
$f_a$ -- допуск на отклонение межосевого расстояния
$$
\Delta \varphi_{л\:\Sigma} = \sum\limits_{j=1}^{N-1} \frac{ \Delta \varphi_{л \: i}}{i_{j - N}} 
$$
\end{document}
