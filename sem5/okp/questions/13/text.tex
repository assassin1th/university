\documentclass{article}
\usepackage{amsmath}
\usepackage{mathtext}
\usepackage[english,russian]{babel}
\usepackage[T2A]{fontenc}
\usepackage[utf8]{inputenc}
\begin{document}
\section{Расчет на точность. Вероятностный метод.}

Допуск на кинематическую погрешность колеса находят как сумму допусков на накопительную
погрешность шага $F_p$ и допуска на погрешность профиля зуба $f_f$:
$$
F_i^{'} = F_p^{'} + f_{f_i}
$$
Допуск на угловую кинематическую погрешность в угловых минутах находят так:
$$
\Delta \varphi_{max, i} = \frac{6.88 \cdot K F_i^{'}}{m z_i} 
$$
Где $K$ - коэффициент фазовой компенсации.

Теперь считаем минимальный допуск на угловую кинематическую погрешность.

\begin{tabular}{cc}
$ \Delta \varphi_{min,i} = \frac{4.88 \cdot K_s F_i^{'}}{m z_i} $ & для 7,8 класса точности \\
$ \Delta \varphi_{min,i} = \frac{4,3 \cdot K_s F_i^{'}}{m z_i} $  & для 5, 6, 9, 10 классов точности
\end{tabular}

Найдем поле рассеивания:
$$
V_i = \varphi_{max, i} - \varphi_{min, i}
$$

Координаты середины поля рассеивания: $E_i = \frac{ \Delta \varphi_{max, i} + \Delta \varphi_{min, i}}{2} $
Суммарная вероятностная кинематическая погрешность:
$$
\Delta \varphi_{i, \Sigma}^{B} = \sum\limits_{j = 1}^{N} \frac{E_{i,j}}{i_j - N} + t_1 \sqrt[2]{\sum\limits_{j = 1}^{N} \left(\frac{V_{i,j}}{i_j - N}\right)^2}
$$
$t_1$ -- коэф-т Стьюдента

\underline{Опр-е погрешностей, вносимых мертвым ходом} 

$$
V_{л, i} = \Delta \varphi_{л, max_j} - \Delta \varphi_{л, min_j}
$$
$$
\Delta \varphi_{л min_j} = \frac{7.33 \cdot j_{n, min}}{m z_j} 
$$
$$
\Delta \varphi_{л max_j} = \frac{7.33 \cdot j_{n, max}}{m z_j} 
$$
$$
E_{л_j} = \frac{ \Delta \varphi_{j \: л \: max} + \Delta \varphi_{j \: л \: min}}{2}
$$
$$
\Delta \varphi_{л}^{Вер} = \sum\limits_{j = 1}^{N} \frac{E_{л \: j}}{i_{j - N}} + t_2 \sqrt[2]{\sum\limits_{j = 1}^{N} \left(\frac{V_{i \: j}}{i_{j - N}} \right)^2}
$$

\end{document}

