\documentclass{article}
\usepackage{amsmath}
\usepackage{mathtext}
\usepackage[english,russian]{babel}
\usepackage[T2A]{fontenc}
\usepackage[utf8]{inputenc}
\begin{document}
\underline{Работа Александра Рябушки, Группа ИУ4-51Б, Вариант №17}

\underline{Преподаватель - Кувшинов Андрей Владимирович}

Отсчетное устройство:

согласно условию:
\begin{itemize}
	\item $x_{max} - x_{min} = \varphi = 210^{\circ}$
	\item $ \Delta x = 0.2^{\circ}$
\end{itemize}

Таким образом:
\begin{enumerate}
	\item Определение цены деления шкалы.
	$$
	b = H = 2 \Delta x
	$$
	$$
	H = |2 \Delta x| = 2 * 0.2^{\circ} = 0.4^{\circ}
	$$
	\item Определение числа делений шкалы:
	$$
	N = \frac{x_{max} - x_{min}}{H} = \frac{210^{\circ}}{0.4^{\circ}} = 525
	$$
	\item Определение длины деления шкалы. Для зрительного определения при нормальных условиях с расстояния 250 - 400 мм рекомендуется брать расстояние между штрихами $b = (1..2.5) мм$. Пусть $b = 2 мм$, тогда:
	\begin{itemize}
		\item Отсчетное устройство с прямолинейной шкалой:	
		$$
		L = 525 * 2 мм = 1050 мм = 1.05 м
		$$
		Что нам не подходит
		\item Отсчетное устройство с круговой или цилиндрической шкалой:
		$$
		D = \frac{L 360}{ \pi \psi} = \frac{1050 * 360}{3.14 * 210} = 573.24 мм
		$$
		Что тоже нам не подходит ввиду слишком большого диаметра цилиндрической шкалы.
		\item Двухшкальный отсчетные устройства:

		Пусть $N_{што} = 50$, тогда:
		$$
		N_{шго} = \frac{N}{N_{што}} = \frac{500}{50} = 10.5
		$$
		Таким образом длина шкал будет равняться:
		$$
		L_{што} = N_{што} b = 100 мм
		$$
		$$
		L_{шго} = N_{шго} b = 21 мм
		$$
		Тогда диаметры счетных дисков:
		$$
		D_{што} = \frac{L_{што} 360}{ \pi \psi_{што}} = \frac{100 * 360}{3.14 * 360} = 31.8309 
		$$
		$$
		D_{шго} = \frac{L_{шго} 360}{ \pi \psi_{шго}} = \frac{21 * 360}{3.14 * 210} = 11.46
		$$
		Необходимо выбрать диаметры из стандартного ряда, тогда пусть $D_{што} = D_{шго} = 35 мм$. Пересчитаем $b$:
		$$
		b_{што} = \frac{D_{што} \pi \psi_{што}}{360 N_{што}} = \frac{35 мм * 3.14 * 360}{360 * 50} = 2.1991 мм
		$$
		$$
		b_{шго} = \frac{D_{шго} \pi \psi_{шго}}{360 N_{шго}} = \frac{35 мм * 3.14 * 150}{360 * 10} = 6.1 мм
		$$
		$ 1 < b_{што} < 2.5 мм$ Что удовлетворяет условию. Коэфициент передачи:
		$$
		i_{што-шго} = \frac{ \varphi_{што}}{ \varphi_{шго}} = \frac{N_{шго} 360}{ \psi_{шго}} = \frac{10 * 360}{210} = 18 
		$$
	\end{itemize}
	\item Расчет ЭМП

	По условию:
	\begin{itemize}
		\item $M_{н} = 0.6 Нм$
		\item $J_н = 0.15 кг * м^2$
		\item $ \omega = 1.7 \frac{1}{с}$
		\item $ \varepsilon = 4 \frac{1}{c^2} $
		\item $L = 1000 ч$
		\item $ \Delta \varphi = 12^{'}$
		\item $t = 0.25 с$
		\item Пуски частые
		\item Метод расчета - вероятностный
	\end{itemize}
	Приступим к расчетам
	\begin{enumerate}
		\item Для начала расчитаем требуемую мощность:
		$$
		P_н = M_н \omega = 1.02 Вт
		$$
		$$
		P_р = \frac{P_н}{ \eta} = \frac{1.02 Вт}{0.8} = 1.275
		$$
		$$
		P_p \zeta_{min} \le P_t \le P_p \zeta_{max}
		$$	
		$$
		1.53 \le P_t \le 3.1875
		$$
		\item (В силу большого числа расчета двигатель был выбран на основании приведенной в конце таблицы, как двигатель с наименьшей номинальной мощностью среди подходящих под условие задачи) Выбираем двигатель ДПР-52-Н1, Н2, Ф1, Ф2-03

		Его характеристики:
		\begin{itemize}
			\item $P_н = 4.6 Вт$
			\item $n_{ном} = 4500 \frac{об}{мин} $
			\item $J_p = 1.7 * 10^{-6} кг * м^2$
			\item $M_{ном} = 9.8 Н * мм$
			\item $M_п = 54 Н * мм$
			\item $U = 27 В$
		\end{itemize}
		\item Проверка двигателя по моментам:
		$$
		M_п \ge M_{ \Sigma}
		$$
		$$
		M_н \ge M_{ст. пр.}
		$$
		Где:
		$$
		M_{ст. пр.} = \frac{M_{ном}}{ \eta i_0} 
		$$
		$$
		M_{дин. пр.} = J_{пр} \varepsilon_н i_0
		$$
		Угловая частота двигателя:
		$$
		\omega_{дв} = \frac{n_{ном} \pi}{30} 
		$$
		Тогда передаточный коэфициент:
		$$
		i_0 = \frac{ \omega_{дв}}{ \omega_{н}} = \frac{n_{ном} \pi}{30 \omega_{н}} = \frac{4500 * 3.14}{30 * 1.7} = 277.05
		$$
		Отсюда:
		$$
		M_{ст. пр.} = \frac{M_н}{ \eta i_0} = \frac{0.6}{0.8 * 277.05 * 10^{-3}} = 2.707
		$$
		Момент инерции приведенный:
		$$
		J_{пр} = \left(1 + K_М\right) J_p + \frac{J_н}{i_0^2} = \left(1 + 1\right) * 1.7 * 10^{-6} + \frac{0.15}{277.05^2} = 5.34 * 10^{-6} кг * м^2
		$$
		Где $K_M = 1$. Момент динамический приведенный:
		$$
		M_{дин. пр.} = J_{пр} \varepsilon_н i_0 = 5.34 * 10^{-6} * 1 * 277.05 = 5.933 н * мм
		$$
		Так как $M_п \ge M_{дин. пр.} + M_{ст. пр.}$ и $M_{ном} \ge M_{ст. пр.}$ проверка на моменты пройдена
		
		\item Проверка на скорость разгона

		Электромеханическая постоянная привода:
		$$
		T_{эм} = \frac{ J_{пр} \omega_{дв}}{M_п - М_{ном}} = \frac{5.34 * 10^{-6} * 4500 * \pi}{\left(54 - 9.8\right) 30 * 10^{-3}} = 0.057
		$$
		Тогда время разгона:
		$$
		t_p = 3 T_{эм} = 2 * 0.0443 = 0.1711
		$$
		Что удовлетворяет поставленному условию, т. е. проверку на скорость разгона прошел.
	\end{enumerate}
	\item Исходя из рассчитанных ранее значений: $i_0 = 277.05$ и $i_{што-шго} = 18$. А так же из условия равногабаритной системы:
	$$
	i_{ост} = \frac{i_0}{i_{што-шго}} = \frac{277.05}{18} = 15.39
	$$
	$$
	n = 1.85 \log_{10}{i_{ост}} = 1.85 \log_{10}{15.39} = 2.19
	$$
	Округляя n в большую сторону до целого получаем $n = 3$. Тогда:
	$$
	i_1 = i_2 = i_3 = \sqrt[3]{i_{ост}} = 2.4875
	$$
	$$
	i_4 = 3
	$$
	$$
	i_5 = 6
	$$
	Подбираем колеса: $z_1 = 28 z_7 = z_9 = 25 z_3 = 30 z_5 = 34$. Тогда:
	$$
	z_2 = i_{12}z_1 = 69.202
	$$
	$$
	z_4 = i_{34}z_4 = 74.145
	$$
	$$
	z_6 = i_{56}z_5 = 84.031
	$$
	$$
	z_8 = i_{78}z_7 = 75
	$$
	$$
	z_{10} = i_{9 10}z_9 = 150
	$$
	Тогда с учетом таблицы 3 (стр.23 методических указаний): $z_2 = 70$, $z_4 = 75$, $z_6 = 85$, $z_8 = 75$, $z_{10} = 150$. Тогда практические коэфициенты передачи:
	$$
	i_{12 пр} = \frac{z_2}{z_1} = 2.5
	$$
	$$
	i_{34 пр} = \frac{z_4}{z_3} = 2.5
	$$
	$$
	i_{56 пр} = \frac{z_6}{z_5} = 2.5
	$$
	$$
	i_{пр} = i_{12 пр} i_{34 пр} i_{56 пр} = 15.625
	$$
	$$
	\zeta = \frac{i_{ост} - i_{пр}}{i_{ост}} 100\% = 1.5\%
	$$
	\item
	Проводим расчет моментов:

	Из методических указаний (стр. 28): $ \eta_{опор} = 0.95$ и $ \eta_{пер} = 0.98$. Тогда:
	$$
	M_{10} = M_{н} = 600 н мм
	$$
	$$
	M_8 = M_9 = \frac{M_{10}}{i_{9 10} \eta_{опор} \eta_{пер}} = 107.411
	$$
	$$
	M_6 = M_7 = \frac{M_{8}}{i_{78} \eta_{опор} \eta_{пер}} = 38,45
	$$
	$$
	M_4 = M_5 = \frac{M_{6}}{i_{56} \eta_{опор} \eta_{пер}} = 16.52
	$$
	$$
	M_2 = M_3 = \frac{M_{4}}{i_{34} \eta_{опор} \eta_{пер}} = 7.09
	$$
	$$
	M_1 = \frac{M_{2}}{i_{12} \eta_{опор} \eta_{пер}} = 3.05
	$$
	\item Расчет модулей. Расчет ведем по формуле для прямозубых цилиндрических колес:
	$$
	m = K_M \sqrt[3]{\frac{M Y_F K}{z \psi_m [ \sigma_F]}}
	$$
	Где, $K_M = 1.4$, $Y_F$ в соот с таблицей 4 стр. 32. $K = 1.5$, $ \psi_m = 10 $.
	Для того чтобы провести расчет модулей необходимо выбрать материалы для зубчатых колес.

	Выбор материала зубчатых колес.

	Поскольку работа редуктора осуществляется при небольших окружных скоростях, $ < 3 \frac{м}{с} $, то в качестве материала выберем сталь 45 из таблицы 7, стр.36. Вид термической обработки - закалка и отпуск.
	Таким образом согласно таблице 10 стр. 40.: $ \sigma_{FR} = 550 МПа$, тогда по формуле:
	$$
	[ \sigma_F] = \sigma_{FR} K_{FC} K_{FL} / S_F = 550 Мпа * 1 * 1 / 2.5 = 220 МПа
	$$
	Таким образом:

	$$
	m_1 = 0.092 мм
	$$
	$$
	m_2 = 0.089 мм
	$$
	$$
	m_3 = 0.119 мм
	$$
	$$
	m_4 = 0.115 мм
	$$
	$$
	m_5 = 0.151 мм
	$$
	$$
	m_6 = 0.146 мм
	$$
	$$
	m_7 = 0.225 мм
	$$
	$$
	m_8 = 0.215 мм
	$$
	$$
	m_9 = 0.317 мм
	$$
	$$
	m_{10} = 0.304 мм
	$$

	Данные для расчета приведены в Приложении А.

	Таким образом согласно таблице 6 стр. 34 и габаритам выходного вала выбранного двигателя:
	$$
	m_1 = m_2 = m_3 = m_4 = m_5 = m_6 = m_7 = m_8 = m_9 = m_{10} = 0.35 мм
	$$
	\item Расчет на контактную прочность
	Поскольку модуль по результатам предыдущих расчетов у всех колес одинаковый, расчет будем вести по выходному 10-му колесу, как по наиболее нагруженному.
	$$
	\alpha = \frac{m_{10} (z_{10} + z_9)}{2} = 30.625 мм 
	$$
	В соответствии с выбранным материалом:
	$$
	[ \sigma_H] = \frac{\sigma_{HR} K_{HL} }{S_{H}} = \frac{ \sigma_{HR} \sqrt[6]{ \frac{10^7}{n c L} }}{S_{H}} = \frac{880 \cdot 1 \cdot 1 \cdot \sqrt[6]{ \frac{10^7}{60 \cdot 2 \cdot \pi \cdot 1.3 \cdot 2 \cdot 3 \cdot 10^3} }}{2.5} = \frac{880 * 1.14}{1.5} = 668,8
	$$
	Тогда:
	$$
	\alpha \ge K_{ \alpha} (1 + i_{9,10})\sqrt[3]{\frac{M_{10} K}{\psi_{ba} i_{9,10}^2 [ \sigma_H]^2}} = 48.5 \cdot (1 + 6) \sqrt[3]{\frac{600 \cdot 1.4}{0.1 6^2 668.8^2}} = 27.32
	$$
	$30.625 > 27.32$ -- Расчет на контактную прочность пройден.

	Данные для расчета приведены в приложении Б.

	\item Расчет геометрических параметров
	\begin{enumerate}
		\item делительный диаметр $d = m z$
		\item Диаметр впадин $d_f = m z - 2 m (h_a^* + c^*)$
		\item Диаметр основной окружности $d_b = d * \cos{ \alpha}$
		\item Диаметр окружности вершин $d_a = m z + 2 m h_a^*$
		\item Расстояние между окружностями вершин и впадин (высота зуба) $h = \frac{d_a - d_f}{2}$
		\item Высота делительной головки зуба $h_a = \frac{d_a - d}{2} $
		\item Высота делительной ножки зуба $h_f = \frac {d - d_f}{2}$
		\item Ширина колеса $b_2 = \psi m$
		\item Ширина шестерни $b_1 = b_2 + 2 m$
		\item Делительное межосевое расстояние $a = \frac{m (z_1 + z_2)}{2} $
	\end{enumerate}
	Результаты вычислений приведены в приложении В.

	\item Расчет на точность.
	Все шестерни производятся с точностью обработки до 6H. Таким образом, Кинематическая погрешность, рассчитывается как:
	\begin{enumerate}
		\item Допуск на кинематическую погрешность колеса находят как: $ F_{i}^{'} = F_p + f_f$, где $F_p$ -- допуск на накопленную погрешность шага, $f_f$ -- допуск на погрешность профиля зуба
		\item Погрешность одной пары зацепления $ \Delta \varphi = \frac{F_{1,max}^{'} + F_{2,max^{'}}}{m z_2} $
		\item Кинематическая погрешность $ \Delta \varphi_{io \Sigma} = \sum\limits_{j=1}^{N} { \frac{ \Delta \varphi_j}{i_{j-N}}} K_{\varphi_j}$
	\end{enumerate}
	Таким образом результаты расчетов по приденным формулам можно увидеть в приложении Г.
	
	Суммарная кинематическая погрешность составила $6.57^{'}$

	Люфтовая погрешность мертвого хода:
	\begin{enumerate}
		\item Максимальное значение мертвого хода находят как:
		$$
		j_{t max} = 0.7(E_{HS1} + E_{HS2}) + \sqrt[2]{0.5(T_{H1}^2 + T_{H2}^2) + 2 (f_a)^2}
		$$
		Где $E_{HS1}$, $E_{HS2}$ -- наименьшее смещение исходного контура шестерни и колеса; $T_{H1}$, $T_{H2}$ -- допуск на смещение исходного контура шестерни и колеса соответственно; $f_a$ -- допуск на отклонение межосевого расстояния передачи;
	\end{enumerate}
	Результаты и исходные данные расчетов приведены в приложении Д.

	С учетом люфтовыбирающего колеса на выходном валу суммарная люфтовая погрешность мертвого хода: $1.93$

	\item Расчет валов и опор.
	Данные и результаты приведены в приложении E.

	Так же расчет на упругий мертвый ход приведен в приложении Ё.

	Суммарная погрешность цепи составила $10.79^{'}$ что проходит по требованиям точности
	
	\item Расчет потенциометра.
	
	Выберем потенциометр ПТП-11 однооборотный. Разрешающая способность $0.1$.
	$$
	\varphi_{пн} \cdot 0.8 \le \varphi_{B} \le \varphi_{п}
	$$
	$$
	285 \le 360 \le 360
	$$
	Подходит.
\end{enumerate}
	
\end{document}
