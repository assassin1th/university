\documentclass{article}
\usepackage{amsmath}
\usepackage{mathtext}
\usepackage[english,russian]{babel}
\usepackage[T2A]{fontenc}
\usepackage[utf8]{inputenc}
\begin{document}
Оформление технического задания (ТЗ)

Конструкторско-технологические расчеты ПП.

Разработка чертежей ПП с помощью САПР: Размещения ЭРИ и трассировка провдников проводящих слоев ПП.

Изготовление ПП по выбранной технологии

Контроль

Испытание

Изучение и анализ технического задания на изделие

Анализ назначения, применения и объекта установки ЭА необходим для определния ограничений и возможностей конструирования ПП. Например для бортовой аппаратуры существуют массогабаритные ограничения, ограничения на элементную базу, варианты установки ЭРЭ и прочее. Высокое быстродействие требует применение материалов с низкой дижлектрической проницаемостью.

По результатам анализа объекта установки необходимо определить:
%Список
\begin{itemize}
	\item для какого уровня модульности будет разрабатываться ПП
	\item унифицированная кострукция пп или нет
	\item К какой группе жесткости и типу объекта установки отностися ЭА (Должен прописан в ТЗ)
\end{itemize}
% Оформить как отступление от темы
\begin{itemize}
	\item Модуль 0-го уровня --- Интегральные схемы
	\item Модуль 1-го уровня --- ПП + установленные компоненты
	\item Модуль 2-го уровня --- ячейка + корпус
\end{itemize}

По объекту установки: 
\begin{itemize}
	\item наземная
	\item морская
	\item бортовая

\end{itemize}
Наземная аппаратура: 
\begin{itemize}
	\item Стационарная --- отсутствие механических перегрузок во время работы, воздух без химических примесей и пыли; подверженность механическим воздействиям в нерабочим состоянии при транспортировке.
	\item Возимая --- работает в условиях вибрации, ударов, абразивной пыли, избыточной влажности. Должна иметь ограниченные габариты и массу, обеспечвать простоту и надежность электрических соединений, устойчивость к ударам и вибрациям, к возникновению инея и росы, а также ограниченную мощность рассеяния.
	\item Носимая --- минимальные габариты и масса, зависимость конструкции от габаритов и массы источников питания, устойчивость к случайным значительным ударам, изменению температур, к конденсации росы, воздействию инея, дождя и пыли.
	\item Бытовая --- повышение технологичности конструкции с целью снижения стоимости, габаритов и массы, простота эксплуатации, массовое производство.
	\item Морская --- 100\% влажность при повышенной температуре и солевом тумане, при непрерывной вибрации от двигателей, ударных перегрузка, линейных ускорениях, акустических, магнитных и радиационных воздействий. Требуется коррозионная стойкость, плеснестойкость, водо- и брызгозащищенность, защищенность от высокочастотных и низкосчастотных электромагнитных полей.
\end{itemize}
Бортовая: 
\begin{itemize}
	\item самолетная
	\item космическая
	\item ракетная
\end{itemize}
Особенности --- постоянный рост функциональной сложности при минимальных габаритах и массе, работа в условиях раряженной амосферы.
Самолетная также испытывает значительные вибрационные, ударные и линейные перегрузки, воздействие перепадов температур, тепловых ударов, если ЭА расположена вне гермоотсеков. Характеризуется кратковременность непрерывной работы, измеряемой часами и длительной преполетной проверкой. Поэтому необходима высокая ремонтопригодность и контролепригодность.
Космическая и ракетная дополнительно требования ограниченности объема и массы, защита от совместного действия вибрационных и линейны нагрузок во время старта, чрезвычайно высокой беотказности, учет специфики больших высот.

Выбор климатических условий

ГОСТ 15150-69 --- использование апаратуры для различных климатических районов

\begin{itemize}
	\item О -- Общеклиматическое исполнение (на суше).
	\item В -- Всеклиматическое исполнение (на суше и на море).
	\item У -- С умеренным климатом.
	\item ХЛ -- С холодным климатом.
	\item УХЛ -- С умеренным и холодным климатом.
	\item ТВ -- С влажным тропическим климатом.
	\item ТС -- С сухим тропически климатом.
\end{itemize}

Вторая цифра кода: категория размещения электрооборудования
\begin{itemize}
	\item 1 Для работы на открытом воздухе.
	\item 2 Для работы в помещениях, где колебания влажности воздуха не очень отличаются от колебаний на открытом воздухе, например: в палатках, прицепах.
	\item 3 Для работы в закрытых помещениях с природной вентиляцией, без искусственного регулирования климатических условия, где колебания темппературы и влажности воздуха, а также действия действие песка и пыли значительно меньше чем снаружи,ветра, атмосферных осадков, отсутствие росы).
	\begin{itemize}
		\item 3.1 Для эксплуатиации в нерегулярно отапливаемых помещениях (объемах)
	\end{itemize}
	\item 4 Для работы в помещниях с искусственно регулируемым микроклиматом, например: в закрытых обогреваемых и вентилируемых производственных и других в том числе подземных, помещениях с хорошей вентиляцией (отсутствие прямого действия атмосферных осадков, ветра, а так же песка и пыли внешнего воздуха).
	\begin{itemize}
		\item 4.1 -- Помещения с кондиционированным или частично кондиционированным воздухом
		\item 4.2 -- Лабораторные, капитальные жилые и другие подобного типа помещения
	\end{itemize}
	\item 5 -- Для работы в помещениях с повышенной влажностью
\end{itemize}
\end{document}
