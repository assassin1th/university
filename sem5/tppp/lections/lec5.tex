\documentclass{article}
\begin{document}
Оформление технического задания (ТЗ)
Конструкторско-технологические расчеты ПП.
Разработка чертежей ПП с помощью САПР: Размещения ЭРИ и трассировка провдников проводящих слоев ПП.
Изготовление ПП по выбранной технологии
Контроль
Испытание

Изучение и анализ технического задания на изделие
Анализ назначения, применения и объекта установки ЭА необходим для определния ограничений и возможностей конструирования ПП. Например для бортовой аппаратуры существуют массогабаритные ограничения, ограничения на элементную базу, варианты установки ЭРЭ и прочее. Высокое быстродействие требует применение материалов с низкой дижлектрической проницаемостью.

По результатам анализа объекта установки необходимо определить:
%Список
для какого уровня модульности будет разрабатываться ПП
унифицированная кострукция пп или нет
К какой группе жесткости и типу объекта установки отностися ЭА (Должен прописан в ТЗ)
% Оформить как отступление от темы
Модуль 0-го уровня --- Интегральные схемы
Модуль 1-го уровня --- ПП + установленные компоненты
Модуль 2-го уровня --- ячейка + корпус

По объекту установки: наземная, морская, бортовая. Вкаждом классе различают специализированные группы.
Наземная аппаратура: стационарная, возимая, носимая, бытовая.
Стационарная --- отсутствие механических перегрузок во время работы, воздух без химических примесей и пыли; подверженность механическим воздействиям в нерабочим состоянии при транспортировке.
Возимая --- работает в условиях вибрации, ударов, абразивной пыли, избыточной влажности. Должна иметь ограниченные габариты и массу, обеспечвать простоту и надежность электрических соединений, устойчивость к ударам и вибрациям, к возникновению инея и росы, а также ограниченную мощность рассеяния.
Носимая --- минимальные габариты и масса, зависимость конструкции от габаритов и массы источников питания, устойчивость к случайным значительным ударам, изменению температур, к конденсации росы, воздействию инея, дождя и пыли.
Бытовая --- повышение технологичности конструкции с целью снижения стоимости, габаритов и массы, простота эксплуатации, массовое производство.
Морская --- 100\% влажность при повышенной температуре и солевом тумане, при непрерывной вибрации от двигателей, ударных перегрузка, линейных ускорениях, акустических, магнитных и радиационных воздействий. Требуется коррозионная стойкость, плеснестойкость, водо- и брызгозащищенность, защищенность от высокочастотных и низкосчастотных электромагнитных полей.
Бортовая: самолетная , космическая, ракетная.
Особенности --- постоянный рост функциональной сложности при минимальных габаритах и массе, работа в условиях раряженной амосферы.
Самолетная также испытывает значительные вибрационные, ударные и линейные перегрузки, воздействие перепадов температур, тепловых ударов, если ЭА расположена вне гермоотсеков. Характеризуется кратковременность непрерывной работы, измеряемой часами и длительной преполетной проверкой. Поэтому необходима высокая ремонтопригодность и контролепригодность.
Космическая и ракетная дополнительно требования ограниченности объема и массы, защита от совместного действия вибрационных и линейны нагрузок во время старта, чрезвычайно высокой беотказности, учет специфики больших высот.

Выбор климатических условий

ГОСТ 15150-69 --- использование апаратуры для различных климатических районов

О -- Общеклиматическое исполнение (на суше).
В -- Всеклиматическое исполнение (на суше и на море).
У -- С умеренным климатом.
ХЛ -- С холодным климатом.
УХЛ -- С умеренным и холодным климатом.
ТВ -- С влажным тропическим климатом.
ТС -- С сухим тропически климатом.

Вторая цифра кода: категория размещения электрооборудования
1 Для работы на открытом воздухе.
2 Для работы в помещениях, где колебания влажности воздуха не очень отличаются от колебаний на открытом воздухе, например: в палатках, прицепах.
3 Для работы в закрытых помещениях с природной вентиляцией, без искусственного регулирования климатических условия, где колебания темппературы и влажности воздуха, а также действия действие песка и пыли значительно меньше чем снаружи,yfghbvth^ d vtnfkkbxtcrb[ c ntgkjbpjkzwbtq? rfvtyys[? ,tnjyys[? lthtdzyys[ gjvtotybz[ (pyfxbntkmyjt evtymitybt ltqcndbz cjkytxyjq hflbfwbb?ветра, атмосферных осадков, отсутствие росы).
3.1 Для эксплуатиации в нерегулярно отапливаемых помещениях (объемах)
4 Для работы в помещниях с искусственно регулируемым микроклиматом, например: в закрытых обогреваемых и вентилируемых производственных и других в том числе подземных, помещениях с хорошей вентиляцией (отсутствие прямого действия атмосферных осадков, ветра, а так же песка и пыли внешнего воздуха).
4.1 -- Помещения с кондиционированным или частично кондиционированным воздухом
4.2 -- Лабораторные, капитальные жилые и другие подобного типа помещения
5 -- Для работы в помещениях с повышенной влажностью
\end{document}
