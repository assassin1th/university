\documentclass{article}
\usepackage{amsmath}
\usepackage{mathtext}
\usepackage[english,russian]{babel}
\usepackage[T2A]{fontenc}
\usepackage[utf8]{inputenc}
\begin{document}
\underline{Получение заготовок ПП} 

Заготовка ПП --- материал основания ПП определенного размера, который подвергается обработке на всех производственных операциях.

Заготовки делятся на:
\begin{itemize}
	\item Единичные -- состоят из площади ПП и технологического поля по периметру.
	\item Групповые -- содержит матрицу однотипных ПП
\end{itemize}

Технологические отверстия:
\begin{itemize}
	\item Базовые отверстия -- необходимы для точноого расположения заготовки в процессе ее обработки.
	\item Технологические отверстия -- отверстия, используемые для механического закрепеления заготовки
	\item Тест-купон -- часть заготовки ПП, служащая для оценки качества изготовления ПП методами разрушающего контроля.
\end{itemize}

Ширина технологического поля единичных заготовок -- 10 мм для ЩПП и ДПП, 30 мм для МПП

В площадь групповой заготовки входят технологическое поле 30мм и полосы материала 10 мм между ячейками матрицы ПП.

Для получения заготовок применяют:
\begin{itemize}
	\item Штамповку -- крупносерийное и массовое производство
	\item Резка -- мелкосерийное, серийное и опытное производство
\end{itemize}
\underline{Штамповка.} 
В качестве рабочего инструемна выступает штамп, закрепелнный на прессе.

При резке заготовка формируется в два этапа. На первом этапе производится разрезка листа диэлектрика на полосы. На втором -- резка полосы на заготовки.

Оборудование:
\begin{itemize}
	\item Роликовые, гильотинные ножницы
	\item Дисковая пила
\end{itemize}

Роликовые ножницы: вращаясь в разные стороын, роликовые диски ножи вдавливаются в материал, осуществаляя его разрезание. За счет трения листа материала и ножей между собой заготовка продвигается по инструменту.

Гильотинные ножницы -- два ножа осуществляют рез материала. Один из них неподвижен. При этом материал фиксируется прижимом сверху полсе установки необходимой длины полосы.

\underline{Получение отверстий}

Способы формирования отверстий в основании ПП:
\begin{itemize}
	\item Пробивка: Из-за низкой степени штампуемости слоистых пластиков применима для ПП 1-2 классов точности для отверсий, не требующих металлизации
	\item Сверление: Наиболее распространенный способ. В серийном и массовом производстве выполняется на станках с числовым программным управлением (ЧПУ). Факторы влияющие на качество сверления:
	\begin{itemize}
		\item Геометрия и материал сверла
		\item Точность позиционирования
		\item Способ закрепления заготовки
		\item Скорость резания (метры в минуты) -- путь проходимый в направлении главного движения наиболее удаленной от оси инструмента точкой режущей кромки в единицу времени (метрах в минуту). Если известны частота вращения сверла и его диаметр, то скорость резания подсчитывают по формуле: $V = \frac{\pi D n}{1000} \frac{м}{мин} $

		Где $V$ -- скорость резания, $D$ -- диаметр сверла, $n$ -- частота сверла, $\pi$ -- постоянное число

		\underline{Правило режимов резания:} чем больше диаметр сверла и чем тверже материал, подлежащий сверлению, тем меньше скорость резания. 
		\item Скорость вращения сверла
		\item скорость подачи и обратного хода
		\item Способ удаления стружки и охлаждения и др.
	\end{itemize}
	\item Лазерное сверление
	\item Фотолиторграфия
	\item Воздействие плазмы
\end{itemize}

Требования к качеству отверстий:
\begin{itemize}
	\item Цилиндрическая форма
	\item Гладкие стенки без заусенцев
	\item Отсутствие разрешния дижлектрика и размазывания наволакивания смолы по стенкам)
\end{itemize}

\underline{Число оборотов режущего интрумента} определяются по формуле:
$$
n = \frac{V * 1000}{\pi D } \frac{об}{мин}
$$

Зная диаметр сверла и материал обрабатываемой детали, по справочным таблицам можно выбрать скорость резания, а по скорости резания и диметру сверла определяем по формуле. число оборотов сверла в минуту. При работе сверлами из углеродистой стали величины скорости резания и подачи следует уменьшать на 30 -- 40\%

\underline{Подача $S$} --- Величина перемещения сверла вдоль оси за один его оборот или за один оборот заготовки. Она измеряется в $ \frac{мм}{об}$ так как сверло имеет две режущие кромки, то подача на одну режущую кромку будет $ \frac{S}{2} $. Всегда выгоднее работать с большой подачей и меньшей скоростью резания; в этом случае сверло изнашивается медленнее. 

Основные узлы сверлильного станка:
\begin{itemize}
	\item Основание -- рама из гранитной плиты
	\item Шпиндель -- высокоскоростной электродвигатель, рабочий вал которого оснащается устройством ддля закрепления режущего инструмента.
	\item Привод координатных перемещений -- Перемещаться может либо заготовка с помощью координатного стола, либо блок сверлильных головок.
	\item Автоматическое устройство для смены инструмента -- инструментальный магазин (дисковый, цепного типа) и устройство смены инструмента, передающее инструмент из магазина в шпиндель и обратно (с манипулятором и без)
\end{itemize}

Система ЧПУ -- компьютеризированное управление обработкой заготовки по созданной заранее специальной программе, в которой все представлено виде кодов.

Команды управляющих программ называют подготовительными или G-функциями/

Пример годирования:
G81 X20 Y30 Z10 R3 F100 - цлкл сверления с указанием координат X,Y глубины сверления и диаметра сверла и скорости подачи.

Для увеличения скорости производства заготовки одной партии ПП собирают в пакеты: укладывают друг на друга по три и более шт. и штифтуют.

При сверлении сверзу и снизу пакетов укладываются листы технологического материала (гетинакс, алюминий и др.) для повышения качества сверления: для исключения отрыва фольги при входе и выходе сверла, уменьшение заусенцев, отвод тепла и др.

Лазерное сверление: воздействие излучения на обрабатываемую заготовку ПП, в результате которого происзодит испарение или взрывное разрушение материала. Может осуществляться двумя способами:
\begin{enumerate}
	\item С использованием специальной металлической маски с отверстиями.
	\item Путем подачи дозированного лазерного излучения импульсами малой длительности в зону формирования отверстий по программе
\end{enumerate}

Чаще всего используется мощный газовый лазер ($C O_2$)

Позволяет получать сквозные отверстия диаметром 40-50 мкм, несквозные отверстия до 25 мкм

Фотолитография --- отверстия получают путем воздействия раствора проявителя на фотодиэлектрик.

Плазменная обрабока --- отверстия получают травлением медной фольги и вскрытый диэлектрик удаляется воздействием плазмы.

Плзама --- состояние вещсетва, при котором атомы лишаются электронной оболочнки в сильном высокочастотном электромагнитном поле, в результате чего образуется свободные радикалы кислорода и фтора.

Эти свободные радикалы разрушают полимерные цепи смол и волокон, образуя гаообращные вещества.

\underline{Подготовка поверхности ПП} 
Поднотовка поверхности осуществляется с целью:
\begin{itemize}
	\item Удаления заусенцев, смолы и механических частиц из отверстий после сверления
	\item получения равномерной шероховатости поврехности, т е придание ей структуры, обеспечивающей прочное и надежное сцепление с другими материалами в процессе обработки
	\item удаления оксидов, масляных пятен, захватов пальцами, пыли, грязи, мелких царапин и пр.
	\item активирования поверхности перед химическим нанесением меди (созданием каталитически активных центров в виде маталлических частиц).
\end{itemize}

Мехническая подготовка поверхности:

В мелкосерийном производстве осуществляется вручную смесью венской извести (смесь окиси кальция) и шлифовального порошка под струей воды.

В крупносерийном и массовм производстве механическую подготовку проводят на модульных линияъ конвеерного типа с дисковыми щетками, на которые подается абразиная суспензия.

Абразив --- твердое мелкозернистое или порошкообразное вещество (кремень, наждак, корунд, карборунд, пемза, гранат), применяемое для шлифовки, полировки, заточки

Суспензия --- смесь веществ, где твердое вещество рапределено в виде мельчийшх частиц в жидком веществе во взвешенном (неосевшем) состоянии.

Абразивные материалы отличаются между собой размером (крупностью) зерен. Размер зерен -- зернистость абразивных материало, определяется размерами сторон ячеек двух сит, через которые просеивают материал.

Зернистость кодируется как № (число в сотых долях мм) или М (число в микрометрах).

Пример: №7 -- 0.07 мм или 70 мкм; М63 - 63

Для механической очиистки чаще всего используется карбид кремния (или карборунд), кристаллический оксид (или эелектрокорунд), пемза (пористое вулканическое стекло)

Шероховатость поверхности --- совокупность неровностей поверхности с относительно малым шагами на базовой длине. Измеряется в микрометрах

Для измерения неровностей используют несколько основных параметров:

$R_a$ -- среднее арифметическое из абсолютных значений отклонений профиля в пределах базовой длины.

$R_z$ -- сумма средних абсолютных значений высот пяти наибольших выступов профиля и глубин пяти наибольшах впадин профиля в пределах базовой длины.

Обозначения шероховатости, одинаковой для части поверхностей изделия, может быть помещено в правом верхнем углу чертежа. Это означает, что все поверхности, на которых на изображении не насены обозначения шероховатости, должны иметь шероховатость, указанную пееред условыным обозначением.

Наиболее широко применяется щеточная очистка абразивными материалами (пемзой или оксидом алюминия) при механическом воздействии нейлоновых щеток по касательной к поверхности печатной платы.

\underline{Рольганг} --- оликовый конвейер -- состоит из группы роликов, оси которых закреплены в раме и вращаются в подшиниках.

Расстоние между роликами должно быть не больше половины длины груза.

Две форсунки высокого давления перемещаются по всей ширине конвейера и подающие воду только снизу печатной платы таким образом, что каждое отверсти, в результате, проходит через центр струи, по крайней мере, единожды.

Плюсы:
\begin{enumerate}
	\item Отсутствие химикатов
	\item Простота очистки сточных вод
	\item Невысокая стоимость оборудования и расходных материалов
\end{enumerate}
Минусы:
\begin{enumerate}
	\item Опасность механического повреждения покрытий
	\item Плохое удаление органических веществ
	\item Образование царапин в направлении движения заготовок
\end{enumerate}

Химическая подготовка поверхности ПП:

Прменяется для обработки отверстий после сверления, очистки слоев МПП перед прессованием, подготовки поверхности нефольгированных диэлектриков. Включает следующие операции:
\begin{itemize}
	\item Химическое обезжиривание дляудаления загрязнений органического происхождения (масел, отпечатков пальцев и пр.)
	\item Микротравление -- для удаления оксидных пленок, создания микрорельефа, удаление наволанивания смолы в отверстиях
	\item Обрабока в антистатике
	\item Промывка
	\item Сушка.
\end{itemize}

Химическое обезжиривание:

Химическое обезжиривание заключается в том, что под воздействием щелочи жиры омыляются, а минеральные масла в присутствии специальных поверхностно-активных веществ образуют эмульсю.

\underline{Омыление} -- означает разложение жиров щелочью с образованием глицирина

\underline{Поверхностно-активные вещества (ПАВ)} --- химические соединения, которые псособны концентрироваться на межфазной поверхности раздела двух сред (тел) и взывать снижение поверхностного натяжения (поверностная активность).

Основное назначение -- вступать посредником для соединения веществ, не вступающих в реакцию между собой.

Простейшее ПАВ -- обычное мыло -- молекула предстваляет собой сферу, один полюс которой -- липофильный (соединяется с жирами), а другой -- гидрофильный (вступает в связь с полекулами воды). То есть, одним концом частица ПАВ прикрепляется к частице жира, а другим концом -- к частицам воды.
\end{document}
