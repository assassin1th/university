\documentclass{article}
\usepackage{amsmath}
\usepackage{mathtext}
\usepackage[english,russian]{babel}
\usepackage[T2A]{fontenc}
\usepackage[utf8]{inputenc}
\begin{document}
\underline{Получение заготовок ПП} 

Заготовка ПП --- материал основания ПП определенного размера, который подвергается обработке на всех производственных операциях.

Заготовки делятся на:
\begin{itemize}
	\item Единичные -- состоят из площади ПП и технологического поля по периметру.
	\item Групповые -- содержит матрицу однотипных ПП
\end{itemize}

Технологические отверстия:
\begin{itemize}
	\item Базовые отверстия -- необходимы для точноого расположения заготовки в процессе ее обработки.
	\item Технологические отверстия -- отверстия, используемые для механического закрепеления заготовки
	\item Тест-купон -- часть заготовки ПП, служащая для оценки качества изготовления ПП методами разрушающего контроля.
\end{itemize}

Ширина технологического поля единичных заготовок -- 10 мм для ЩПП и ДПП, 30 мм для МПП

В площадь групповой заготовки входят технологическое поле 30мм и полосы материала 10 мм между ячейками матрицы ПП.

Для получения заготовок применяют:
\begin{itemize}
	\item Штамповку -- крупносерийное и массовое производство
	\item Резка -- мелкосерийное, серийное и опытное производство
\end{itemize}
\underline{Штамповка.} 
В качестве рабочего инструемна выступает штамп, закрепелнный на прессе.

При резке заготовка формируется в два этапа. На первом этапе производится разрезка листа диэлектрика на полосы. На втором -- резка полосы на заготовки.

Оборудование:
\begin{itemize}
	\item Роликовые, гильотинные ножницы
	\item Дисковая пила
\end{itemize}

Роликовые ножницы: вращаясь в разные стороын, роликовые диски ножи вдавливаются в материал, осуществаляя его разрезание. За счет трения листа материала и ножей между собой заготовка продвигается по инструменту.

Гильотинные ножницы -- два ножа осуществляют рез материала. Один из них неподвижен. При этом материал фиксируется прижимом сверху полсе установки необходимой длины полосы.

\underline{Получение отверстий}

Способы формирования отверстий в основании ПП:
\begin{itemize}
	\item Пробивка: Из-за низкой степени штампуемости слоистых пластиков применима для ПП 1-2 классов точности для отверсий, не требующих металлизации
	\item Сверление: Наиболее распространенный способ. В серийном и массовом производстве выполняется на станках с числовым программным управлением (ЧПУ). Факторы влияющие на качество сверления:
	\begin{itemize}
		\item Геометрия и материал сверла
		\item Точность позиционирования
		\item Способ закрепления заготовки
		\item Скорость резания (метры в минуты) -- путь проходимый в направлении главного движения наиболее удаленной от оси инструмента точкой режущей кромки в единицу времени (метрах в минуту). Если известны частота вращения сверла и его диаметр, то скорость резания подсчитывают по формуле: $V = \frac{\pi D n}{1000} \frac{м}{мин} $

		Где $V$ -- скорость резания, $D$ -- диаметр сверла, $n$ -- частота сверла, $\pi$ -- постоянное число

		\underline{Правило режимов резания:} чем больше диаметр сверла и чем тверже материал, подлежащий сверлению, тем меньше скорость резания. 
		\item Скорость вращения сверла
		\item скорость подачи и обратного хода
		\item Способ удаления стружки и охлаждения и др.
	\end{itemize}
	\item Лазерное сверление
	\item Фотолиторграфия
	\item Воздействие плазмы
\end{itemize}

Требования к качеству отверстий:
\begin{itemize}
	\item Цилиндрическая форма
	\item Гладкие стенки без заусенцев
	\item Отсутствие разрешния дижлектрика и размазывания наволакивания смолы по стенкам)
\end{itemize}

\underline{Число оборотов режущего интрумента} определяются по формуле:
$$
n = \frac{V * 1000}{\pi D } \frac{об}{мин}
$$

Зная диаметр сверла и материал обрабатываемой детали, по справочным таблицам можно выбрать скорость резания, а по скорости резания и диметру сверла определяем по формуле. число оборотов сверла в минуту. При работе сверлами из углеродистой стали величины скорости резания и подачи следует уменьшать на 30 -- 40\%

\underline{Подача $S$} --- Величина перемещения сверла вдоль оси за один его оборот или за один оборот заготовки. Она измеряется в $ \frac{мм}{об}$ так как сверло имеет две режущие кромки, то подача на одну режущую кромку будет $ \frac{S}{2} $. Всегда выгоднее работать с большой подачей и меньшей скоростью резания; в этом случае сверло изнашивается медленнее. 

Основные узлы сверлильного станка:
\begin{itemize}
	\item Основание -- рама из гранитной плиты
	\item Шпиндель -- высокоскоростной электродвигатель, рабочий вал которого оснащается устройством ддля закрепления режущего инструмента.
	\item Привод координатных перемещений -- Перемещаться может либо заготовка с помощью координатного стола, либо блок сверлильных головок.
	\item Автоматическое устройство для смены инструмента -- инструментальный магазин (дисковый, цепного типа) и устройство смены инструмента, передающее инструмент из магазина в шпиндель и обратно (с манипулятором и без)
\end{itemize}

Система ЧПУ -- компьютеризированное управление обработкой заготовки по созданной заранее специальной программе, в которой все представлено виде кодов.

Команды управляющих программ называют подготовительными или G-функциями/

Пример годирования:
G81 X20 Y30 Z10 R3 F100 - цлкл сверления с указанием координат X,Y глубины сверления и диаметра сверла и скорости подачи.

Для увеличения скорости производства заготовки одной партии ПП собирают в пакеты: укладывают друг на друга по три и более шт. и штифтуют.

При сверлении сверзу и снизу пакетов укладываются листы технологического материала (гетинакс, алюминий и др.) для повышения качества сверления: для исключения отрыва фольги при входе и выходе сверла, уменьшение заусенцев, отвод тепла и др.

Лазерное сверление: воздействие излучения на обрабатываемую заготовку ПП, в результате которого происзодит испарение или взрывное разрушение материала. Может осуществляться двумя способами:
\begin{enumerate}
	\item С использованием специальной металлической маски с отверстиями.
	\item Путем подачи дозированного лазерного излучения импульсами малой длительности в зону формирования отверстий по программе
\end{enumerate}

Чаще всего используется мощный газовый лазер ($C O_2$)

Позволяет получать сквозные отверстия диаметром 40-50 мкм, несквозные отверстия до 25 мкм

Фотолитография --- отверстия получают путем воздействия раствора проявителя на фотодиэлектрик.

Плазменная обрабока --- отверстия получают травлением медной фольги и вскрытый диэлектрик удаляется воздействием плазмы.

Плзама --- состояние вещсетва, при котором атомы лишаются электронной оболочнки в сильном высокочастотном электромагнитном поле, в результате чего образуется свободные радикалы кислорода и фтора.

Эти свободные радикалы разрушают полимерные цепи смол и волокон, образуя гаообращные вещества.

\underline{Подготовка поверхности ПП} 
Поднотовка поверхности осуществляется с целью:
\begin{itemize}
	\item Удаления заусенцев, смолы и механических частиц из отверстий после сверления
	\item получения равномерной шероховатости поврехности, т е придание ей структуры, обеспечивающей прочное и надежное сцепление с другими материалами в процессе обработки
	\item удаления оксидов, масляных пятен, захватов пальцами, пыли, грязи, мелких царапин и пр.
	\item активирования поверхности перед химическим нанесением меди (созданием каталитически активных центров в виде маталлических частиц).
\end{itemize}

Мехническая подготовка поверхности:

В мелкосерийном производстве осуществляется вручную смесью венской извести (смесь окиси кальция) и шлифовального порошка под струей воды.

В крупносерийном и массовм производстве механическую подготовку проводят на модульных линияъ конвеерного типа с дисковыми щетками, на которые подается абразиная суспензия.

Абразив --- твердое мелкозернистое или порошкообразное вещество (кремень, наждак, корунд, карборунд, пемза, гранат), применяемое для шлифовки, полировки, заточки

Суспензия --- смесь веществ, где твердое вещество рапределено в виде мельчийшх частиц в жидком веществе во взвешенном (неосевшем) состоянии.

Абразивные материалы отличаются между собой размером (крупностью) зерен. Размер зерен -- зернистость абразивных материало, определяется размерами сторон ячеек двух сит, через которые просеивают материал.

Зернистость кодируется как № (число в сотых долях мм) или М (число в микрометрах).

Пример: №7 -- 0.07 мм или 70 мкм; М63 - 63

Для механической очиистки чаще всего используется карбид кремния (или карборунд), кристаллический оксид (или эелектрокорунд), пемза (пористое вулканическое стекло)

Шероховатость поверхности --- совокупность неровностей поверхности с относительно малым шагами на базовой длине. Измеряется в микрометрах

Для измерения неровностей используют несколько основных параметров:

$R_a$ -- среднее арифметическое из абсолютных значений отклонений профиля в пределах базовой длины.

$R_z$ -- сумма средних абсолютных значений высот пяти наибольших выступов профиля и глубин пяти наибольшах впадин профиля в пределах базовой длины.

Обозначения шероховатости, одинаковой для части поверхностей изделия, может быть помещено в правом верхнем углу чертежа. Это означает, что все поверхности, на которых на изображении не насены обозначения шероховатости, должны иметь шероховатость, указанную пееред условыным обозначением.

Наиболее широко применяется щеточная очистка абразивными материалами (пемзой или оксидом алюминия) при механическом воздействии нейлоновых щеток по касательной к поверхности печатной платы.

\underline{Рольганг} --- оликовый конвейер -- состоит из группы роликов, оси которых закреплены в раме и вращаются в подшиниках.

Расстоние между роликами должно быть не больше половины длины груза.

Две форсунки высокого давления перемещаются по всей ширине конвейера и подающие воду только снизу печатной платы таким образом, что каждое отверсти, в результате, проходит через центр струи, по крайней мере, единожды.

Плюсы:
\begin{enumerate}
	\item Отсутствие химикатов
	\item Простота очистки сточных вод
	\item Невысокая стоимость оборудования и расходных материалов
\end{enumerate}
Минусы:
\begin{enumerate}
	\item Опасность механического повреждения покрытий
	\item Плохое удаление органических веществ
	\item Образование царапин в направлении движения заготовок
\end{enumerate}

Химическая подготовка поверхности ПП:

Прменяется для обработки отверстий после сверления, очистки слоев МПП перед прессованием, подготовки поверхности нефольгированных диэлектриков. Включает следующие операции:
\begin{itemize}
	\item Химическое обезжиривание дляудаления загрязнений органического происхождения (масел, отпечатков пальцев и пр.)
	\item Микротравление -- для удаления оксидных пленок, создания микрорельефа, удаление наволанивания смолы в отверстиях
	\item Обрабока в антистатике
	\item Промывка
	\item Сушка.
\end{itemize}

Химическое обезжиривание:

Химическое обезжиривание заключается в том, что под воздействием щелочи жиры омыляются, а минеральные масла в присутствии специальных поверхностно-активных веществ образуют эмульсю.

\underline{Омыление} -- означает разложение жиров щелочью с образованием глицирина

\underline{Поверхностно-активные вещества (ПАВ)} --- химические соединения, которые псособны концентрироваться на межфазной поверхности раздела двух сред (тел) и взывать снижение поверхностного натяжения (поверностная активность).

Основное назначение -- вступать посредником для соединения веществ, не вступающих в реакцию между собой.

Простейшее ПАВ -- обычное мыло -- молекула предстваляет собой сферу, один полюс которой -- липофильный (соединяется с жирами), а другой -- гидрофильный (вступает в связь с полекулами воды). То есть, одним концом частица ПАВ прикрепляется к частице жира, а другим концом -- к частицам воды.

Микротравление:

Главной целью очистки отверстий печатных плат после сверления является удаление эпоксидной смолы со стенок отверстий и придание равномерной шероховатости стенким отверстий. Этапы удаления:
\begin{itemize}
	\item Размягчение эпоксидной смолы в отверстиях. Для этих целей можно использовать сильно концентрированную серную кислоту.
	\item Удаление смолы из отверстия и подтравливание волокон стеклоткани. В данном случае используется перманганат калия $KMnO_4$ или натрия (соли марганцовой кислоты (марганцовка в быту)).
	\item Нейтрализация пеманганата. Этот процесс необходим для того, чтобы нейтрализовать и удалить остатки перманганата из просверленных отверстий.
\end{itemize}

Плюсы химической отчистки:
\begin{itemize}
	\item Отсутствие механического загрязнения поверхности и отверстий
	\item Отсутствие царапин, поверхностных деформаций и напряжений
	\item Улученная адгезия по сравнению с механической очисткой.
\end{itemize}
Минусы химической отчистки:
\begin{itemize}
	\item Неравноемрное или неполное удаление защитных покрытий
	\item Чрезмерное удаление металла с поверхности
	\item Высокие расходы на очистку сточных вод.
\end{itemize}

Комбинированная подготовка поверхности --- химико-механическия подготовка перед химическим меднением: механическая очистка, химическая очистка, линия активирования, промывка в холодной воде.

Электохимическая подготовка поверхности: обработка внутренних слоев МПП. Этапы: электрохимическое обезжиривание, декапирование, пассивирование.

\underline{Электрохимическое обезжиривание} --- обезжиривание в растворе, через который пропускают постоянных эдектрический ток ( процесс электролиза).

\underline{Электролит} --- раствор или расплав солей, способный проводить электрический ток вследствие процесса диссоциации -- распада на ионы ( из-за химического взаимодействия с растворителем или процесса плавления).

Упорядоченное движение ионов в электролитах происходит в электричеом поле, которое создаётся \underline{электродами} --- проводниками, соединёнными с полюсами источника электрической энергии. \underline{Катодом}  при эелктролизе наызвается отрицательный электрод, \underline{анодом}  -- положительный. Положительные ионы --- \underline{катионы}, отрицательные ионы --- \underline{анионы} -- движутся к аноду.

Существуют два варианта очистки:
\begin{enumerate}
	\item При катодном обезжиривании деталь выступает в качестве катода.
	При воздействии тока ионы водорода восстанавливаются на катоде, происходит выделение водорода в виде пузырьков. В результате жиры механически удаляются, происходит отрыв капель масел вместе с пузырьками газа.

	Недостаток -- поверхность изделий сильно насыщается  водородом и изделия становятся более хрупкими.
	\item Анодное обезжиривание --- деталь выступает в качестве анода, выделяется кислород. Кислород удаляет жиры с поверхности медленнее, но при этом изделие не делается хрупким. В результате главный минус -- большая длительность процесса.

На практике очень часто применяют их одновременно, т. е. сначала изделия подвергают катодному обезжириванию, затем в конце --- анодному, достигая таким образом лучших результатов.
\end{enumerate}
\underline{Декапирование} --- удаление окислов (оксидной пленки) с поверхности металлических деталей с помощью слабых растворов кислот и щелочей. обычно используется раствор серной или соляной кислоты.

Приандоном декапировании происходит механиечкое отрывание окислов выделяющимся кислородом.

\underline{Пассивирование} --- переход поверхности металла в неактивное, пассивное состояние, связанное с образованием тонких поверхностных слоёв соединений, препятствующих коррозии.

Пассивирование всегда связано с окислительным процессом либо с непосредственным воздействием окислителей, как кислород, озон, крепкая азотная кислота, соли хромовой кислоты, перекись водорода, либо с анодоной поляризацией, что также связано с действием кислорода, выделяющегося на аноде.

Для пассивации меди используют хроматные растворы (соли хромовой кислоты $H_2CrO_4$), при обработке образуется бесцветная пленка.

Плазмохимическое травление поверхности ПП и отверстий:

Применяется для очитки от смолы и сткловолокна отверстий диаметро менее 0.3мм после сверления
Под воздействием плазмы происходит испарение смолы, находящейся на стенках отверстий.

Травление проводится в вакуумной камере в среде кислород-фреон, время  в среднем 45 мин

плюсы:
\begin{itemize}
	\item Очистка отверсий без подтравливания диэлектрика, не требуется очистка сточных вод.
\end{itemize}

Минусы:
\begin{itemize}
	\item Низкая произсовдительность, высокая стоимость оборудования.
\end{itemize}

Фотолитография

\underline{Фотолитография} --- Способ покрытия заданной конфигурации с помощью фоторезистов.

Основные этапы:
\begin{enumerate}
	\item Нанесение фоторезиста
	\item Совмещение с фотошаблоном
	\item Экспонирование
	\item Проявление
\end{enumerate}

\underline{Фоторезист}  --- полимерный светочувствительный материал, т. е. выскомолекулярные соединения (состоят из мономерных свеньев, соединенных в длинные макромолекулы), которые под дейсвием электромагнитного излучения (в том числе видимиого света) меняют свои структурные или физиком-химические свойства. По реакции на свет фоторезисты делятся на негативные и позитивные

Основные свойства фоторезистов:
\begin{enumerate}
	\item Светочувствтельность --- способность изменять свои свойства под воздействием излучения различного спектра: различают фоторезисты для видимой области спектра (380 - 760 нм), ближнего (320 - 450 нм) и дальнего УФ излучнеия (180-320 нм)
	\item Контрастность --- степень четкости границы засвеченных и незасвенченных учатков.
	\item Разрешающая способность $R$ --- максимально возможное воличество линий, разделенных промежутками такой же ширины на 1мм поверхности. $R = \frac{1000}{2L} $, где $L$ -- ширина линии, мкм.
	\item Кислотостойкость --- способность защищать поверхность от воздействия кислот и щелочей, выражается временем до возникновения разрушений.
	\item Одногородномть и равномерность по толщине.
\end{enumerate}

По способу нанесения фоторезисты делятся на:
\begin{itemize}
	\item Сухие плёночные
	\item Жидкие на водных и органических растворителях.
\end{itemize}

Сухие пленочные фоторезисты (СПФ) применяютпри выскоих требованиях по рарешающей способности (воспроизводимые линии 0.04 ... 0.1 мм) и в случаях, когда необходимо избежать попадания фоторезиста в отверстия

СПФ Состоит из трёх слоев: защитной полиэтиленовой пленки, чувствительного к УФ излучению среднего слоя и оптически просрачной лавсановой ленки, предназначенной для защиты от окисления на воздухе

СПФ наносят на ПП на установках ламинаторах. Накатка осуществляется валиком, нагретым до 100..120 градусов. Полиэтиленовая пленка при этом удаляется и наматывается бобину, а лавсановая лпенка остается на поверхности.

Жидкие фоторезисты --- содержат $60-90\%$ органического растворителя

Способы нанесения:
\begin{enumerate}
	\item Центрифугирование -- ФР наностися дозатором затем растекается под дейсвием центробежной силы (минус: неравномерность толщины слоя из-за утолщения по краям).
	\item Окунание или вытягиванеие --- заготовка погружается вертикально в ванну с раствором фоторезиста и поднимается с постоянно скоростью (15...50 $ \frac{см}{мин} $). Толщина ФР зависит от скорости вытягивания и вязкости; равномерность -- от плавности изалечения
(плюсы: двустороннее нанесение.)
	\item Пульверизация (распыление) --- струя фоторезиста поступает из распылительной формунки под углом, близким к $90$, на поверхность заготовки, которая перемещается на конвейере. (Плюсы: простота способа и возможность применения для любых типов ПП. Минусы: неравномерность толщины гна больших поверхностях.)
	\item Электростатическое распыление --- распыление заряженных частиц ФР в поле выского напряжения. (плюсы: высокая равномерность тощны, возможность формирования ноких слоев. минусы: высокая стоимость оборудования и спецаильных фоторезистов)
\end{enumerate}

Общие минусы жидких фоторезистов:
\begin{itemize}
	\item Низкая разрешающая способность
	\item Разрастание првоодников над фоторезистом при гальваническом осаждении меди
	\item Окисление
	\item Зависимость свойств от влажности и температуры
	\item Заполнение отверстий
	\item Низкая устойчивость к электролитам гальванического меднения.
\end{itemize}

Общие плюсы жидких фоторезистов:
\begin{itemize}
	\item Простота применения
	\item Защита боковых поверхностей
	\item Нетоксичность
\end{itemize}

Экспонирование
Экспонирование --- воздействие на светочувствительный материал излучением определенного спектра в течение заданного времени.

Время экспонирования завсит от толщины ФР

Для засветки используются ртутные или ртутно-кварцевые лампы.

Для выборочного воздействия на слой фоторезиста перед экспонированием заготовку совмещают с фотошаблоном.

Фотошаблон для ПП --- полимерная плека со свформированным на её поверхности рисунком из материала ( фотографическая эмульсия, металлическая пленка, окись железа), не пропускающего конкретный спектр излучения.

При экспонировании имеют место дифрация, преломление и отражение света, приводящие к разбросу размеров Элементов рисунка и расмытости их краев.

Основным требованием к выполнению операции экспонирования является получение обходимой степени полимеризации (для негативных ФР) или деструкции (для позитивных ФР), т. е. проработки фотослоя на полную глубину. Это зависит от количесва света $H$, поглощенного ФР, которое определяется освещенностью поверхности $E (лк)$ и верменем обработки $t$. $H = E t$

Люкс (лк) -- единица измерения освещенности, равен освещённости поверхнсти площадью $1 м2$.

При проведении экспонирвоания необъодимо обеспечить:
\begin{itemize}
	\item Плотный и равномерный прижим фотошаблона к поверхности заготовки с помощью вакуума для исключения попадания света под темные участки ФШ.
	\item Равномерное распределение освещенности по поверхности (для этого применяют спсециальные устрйоства, коллиматоры, для получения пучков параллельных световых лучей)
	\item Правильное определение времени экспонирования.
\end{itemize}

Проявление рисунка.

Проявление --- избирательное растворение участков фоторезиста: неэкспонированных для нгетивных и экспонированных для позитивных ФР.

Проявилетями позитивных фоторезистов являются слабые водные и водно-глицериновые щелочные растворы $KOH, NaOH$.

Проявителями негативынх фоторезистов являются органические растворители: безнол, толуол, трихлорэтилен и др.

Скорость проявления зависит от концентрации проявителя, времени экспонирования, наличия перемешивания, температуры.

С ростом температуры скорость проявления растет. Увеличение времени экспонрования уменьшает время провяления.

Основным требованием является четкость контура. Нерезкий контур наызвается вуалью.

Проявление осуществляют погружением в расвтор, выдержкой в парах проявителя или распылением на вращающуюся подложку.

После проявления следует операция тщательной промывки подложек в протоке деионизованной воды.

Для негативных ФР режимы проявления незначительно влияют на качество. Для позитивных же несоблюдение режимов может привести к подтравливанию по контуру незасвеченных участков, что называется перепроявлением.

Задубливание рисунка

Задубливание --- сушка ФР --- проводится для повыешения химической стойкости фоторезиста к травителю и улучшения адгезии фоторезиста к подложке.

При повышенных температурах в слое фоторезиста затягиваются мелкие отверстия, поры, дефекты.

Для большинства фоторезистов максимальная температура задубливания -- $150$, общее время $1-1.5$ часа.

Более высокие температруы выхывют разрушение слоя фоторезиста: поверхность покрывается мелкими терщинами и рельеф полностью теряет свои защитные свойства.

Металлизация ПП

НАзначение процесса метализации --- получение токопроводящих участков ПП (проводников, металлизированных отверстий и др.)

\underline{Химическое меднений} 

Химическое меднение применяют:
\begin{enumerate}
	\item Для получения тонкого (3-5 мкм) подслоя меди на стенках монтажных и переходных отверстий, чтобы сделать их диэлектрические покрытия токороводящими;
	\item Для формирования проводящего рисунка нефольгированных диэлектриках (толщина проводников порядка 35 мкм);
\end{enumerate}

Химическое меднение --- окислительно-восстановительный автокаталитический процесс, т.е. катализатором является один из продуктов химической реакции.

Окислительно-восстановительный процесс --- изменение степени окисления атомов, путем перераспределения электронов между атомом-акцептором (окислителем) и атомом-донором (восстановителем)

На начальном этапе катализатором для выделения меди является металлический палладий, а затем осажденные кристаллы меди катализируют дальнейшее осаждение, и процесс протекает самопроизвольно.

Сначала погружается плата в раствор соли олова. Для того чтобы зафиксировать палладий на стенках. Первая ванна -- соль олова. Вторая ванна -- соль палладия.Третья ванна -- соль меди.

Химическое меднение производят на линиях конвеерного типа, сстоящих из нескольких ванн с растворами солей металлов.

Заготовки ПП Устанавливают при помощи технологических отверстий или зажимов в подвески для химического меднения, изготовляенные например из коррозионной стали или фторопласта.

Затем заготовки полностью погружаются последовательно в каждую из ванн на определенное время.

Перемещение заготовок в массовм производстве выполняется по программе автооператором.

3 этапа:
\begin{enumerate}
	\item Сенсабилизация --- подготовка поверхности ддиэлектрика.
	\item Активированне --- создание на дижеетрике каталитических частиц.
	\item Осаждение меди.
\end{enumerate}
\underline{Адсорбция} --- повешине концентрации компонента в поверхностном слое вещества (на гранцие раздеа фаз) по сравненнию с ее значением в каждой объемной фазе.

Один из используемых составов растворов для меднения:
\begin{itemize}
	\item Медь сернокислая 100 $ \frac{г}{л} $
	\item Натрий едкий 100 $ \frac{г}{л} $
	\item НАтрий улекислый (безводный) 30 $ \frac{г}{л} $
	\item Глицерин 100 $ \frac{г}{л} $
	\item Формалин (33\%) $25-35 \frac{мл}{л} $
\end{itemize}

Скорость процесса химического меднения в завистимости от раствора -- 3-4 $ \frac{мкм}{ч} $

Ванна для химического меднения имеет устройства поддержания температуры, перемешивания равствора, возврано-поступательно перемещения заготовоек для прокачивания раствора через отверстия, чтобы обеспечить полное покрытие стенок медью

Температура обычно задается от 18 до 25 градусов

Один из способов перемешивания -- барботирование -- перемешивание раствора воздухом.

Гальваниеческая металлизация

Гальваническое осаждение покрытий производится в ванная с электролитом под воздействием электрического поля. заготовки, закрепленные в подвесах, полностью погружаются в электролит. Подвес должен быть из проводящего материала (например титан), он закрепляется на специальной штанге . которая подключается к внешнему источнику постоянного тока.

ПП --- катод.

Гальваническое меднение. Скорость осаждения меди зависит от плотности тока.

Аноды изготавливаются из осаждаемого металла или сплава, помещают в чехлы для исключения поппадания продуктов распада анода для заготовки ПП и размещают по обе стороны на одинаковом расстоняии от катода.

Площадь анодов обычно в 1.5-2 раза больше площади заготовк ПП для улучшения равномерности осаждаемого покрытия.

В качестве электролитов используют водны растворы солей осаждаемого металла, который содержится в виде положительно заряженным ионов. Дополнительные ионы металла образуются на аноде при окислительной реакции.

Под действием эл. поля ионы перемещаются к катоду и ваосстанавливаются до нейтральных атомв.

Гальваническим методом можно осадить не только медь, но и олово, которое используется как металло-резист.

Гальваниечкое покрыте должно быть:

\begin{itemize}
	\item Сплошным, без пор, вздутий, темных пятен;
	\item Пластичным, чтобы обеспечить устойчивость к вибрациям, перегибам, короблению ПП.
	\item Равномерным по толлщине на поверхности и в отверстиях ПП.
\end{itemize}

Для пластичности меди в электролит добавляют специальные добавки.

Равномерность зависит от:

\begin{itemize}
	\item Расстояниея между анодом и катодом --- чем больше расстояние тем более равномерное покрытие на заготовках и в отверстиях ПП
	\item Соотношения между площадью анода и катода (анод должен быть в 1.5...2 раза больше катода, чтобы уменьшить осаэдение металла на острых кромках ПП: краях углах и пр.)
	\item Габаритов ПП: чем больше площадь ПП, тем больше неравномерность покрытий;
	\item Плотности тока: чем выше плотность тока, тем больше неравномерность покрытия и наоборот;
	\item Рассеивающей способности электролита, количесвенной характеристикой которой является отношение толщины покрытия в центре отверстия к толщине на поверхности $\left( \frac{H_2}{h_1} \right)$
	\item Соотношение между диаметром отверстия и толщиной ПП;
	\item температуры электролита;
	\item Объема электролита, проходящего через отверстия и пр.
\end{itemize}

Нанесение финишных покрытий.

Для сохранения паяемости печатных плат необходимо защищать медную поверхность контактных площадок финишных металлическим защитным покрытием. На элементы рисунка одной платы, имеющие различное назначение, могут наноситься разные покрытия, если это необходимо для обеспечения качества и надежности. Варианты финишных покрытий:
\begin{itemize}
	\item Лужение свинцовыми или бессвинцовыми припоями;
	\item Химический никель.имеерсионное золото;
	\item Иммерсионное серебро
\end{itemize}

На текущий момент металлическому покрытию предпочитают паяльную маску. Паяльная маска --- защита меди на тех участках, которые незадействованы в контакте с выводами компонентов. Покрытие контактов должно обеспечивать паяемость.

Сплав Вуда --- состав: Висмут 50\%, свинец 20\%, кадмий 12.5\%, олово 12.5\%

Одна из самых низких температур плавления, которая ниже ста градусов Цельсия, что позволяет применять для чувствиетльных к температуре деталей, а так же использовать при этом инструменты малой мощности.

Температура плавления -- 72 градуса цельсия.

Минусы:
\begin{itemize}
	\item Не выдерживает механические нагрузки
	\item Чувствителен к температуре (может все отпаяться)
\end{itemize}

Сплав Розе --- состав: 50\% Висмут, свинец 25\%, олово 25\%. Припой марки ПОСВ-50

Плюсы:
\begin{itemize}
	\item Температура плавления --- 94 градуса по цельсию
\end{itemize}

Минусы:
\begin{itemize}
	\item Хрупкий, кристаллизуется
\end{itemize}

Многослойные печатные платы

Структура многослойных плат

Типы слоев МПП:
\begin{itemize}
	\item Сигнальные слои --- слой разводки првоодников
	\item Слой земли и питания --- проводящий рисунок предназначен только для подачи напряжения питания или заземления
	\item Экранные слои --- металлический слой, обеспечивающий электромагнитную защиту различных цепей.
\end{itemize}

Прессование МПП

Назначение процесса прессования --- соединение отдельных сигнальных слоев на диэлектрическом основании, слоев земли и питания и экранных слоев в монолитную конструкцию при помощи склеивающих прокладок (препрегов).

Сущность процесса --- одновременное прессование в пресс-форме всех проводящих слове, проложенных степлотканью, пропитанной недополимеризированной термореактивной смолой.

Термореактивная смола --- тип полимера, который под действием нагрева и дваления затвердевает, теряя свою плстичность.

\underline{Помелиризация}  --- химическая реакция образованная маромолекул путем последовательго присоединения молекул низкомолекулярных веществ (мономеров) к концам цепей растущих макромолекул.

Для начала полимеризации необходимо ее инициирование --- превращение небольшой доли молекул момномера в активные центры, способные присоединять к себе новые молекулы мономера. Для этого в систему вводят спецальные вещества (инициаторы или катализаторы полимеризации).

Рост цепи макромолекулы происзодит путем многократного повторорения однотипных реакций присоединеня молекул мономера к активному центру $M^*$:
$$
M^* + M \to M_2^*; M_2^* + M \to M_3^*, \dots , M_n^* + M \to M_{n+1}^*
$$

\underline{Процесс полимерицазии смол} 

Смола меняет свое состоянии с жидкого на липкое вязкое гелеобразное. После гелеобразования скорость реакции замедляется по мере нарастания твердости. В твердых телах химические реакции протекают медленнее. От состояния мягкого липкого геля смола переходит к более твердому, постепенно теряя липучесть. Со временем липучесть исчезет и смола продолжит набирать твердость и прочность.

Для прессования МПП применяют многоярусные гибравлиечкие прессы.

Пресс-форма состоит из двух стальных плит; в нижней плите имеется несколько штырей для совмещения слоев и фиксации пакета МПП рпи прессовании. На штырях пресс-формы осуществляется сборка пакета МПП.

Цикл прессования состоит из следуюиз этапов:

\begin{itemize}
	\item $ab$ --- нагрев пакета до температуры прессования $T = 175 +- 5$ при низком давлении (давление первой ступени) 0.05..0.1 МПа. Датчики котроля температуры расположены в плитах пресса.
	\item $bd$ --- приложение высокого давления (вторая ступень), порядка 2-3 МПа, в момент стабилизации периода гелеобразования смолы. Объемное сопротивление прокладочной ткани снимается с датчиков, расположенных на технологическом поле проводящих слоев. Точка $b$ соответствует полному расплавлению смолы --- минимальное объемное сопротивление диэлектрика.
	\item $de$ --- Задержка при температуре полимеризации смолы. $T = 185-200$ в зависимости от типа смолы.
	\item $ef$ --- Охлаждение до температуры $30-40$ под давление (для снятия внутренних напряжений материалов Из-за различных ТКЛР, наличия воздуха и влаги в порах склеивающих прокладок)
\end{itemize}

В настоящее время применяют также автомиатические машины сборки пактов МПП по реперным знакам с автоматической оптической системой и фиксацией по периметру мягкими медными заклепками, что позволяет проводить прессование без пресс-форм.

\underline{Реперный знак}  --- элемент рисунка ПП, который предоставляет общею измерительную точку при позиционированнии. При автоматическом оптическом распознавании происходит вычисление необходимого смещения заготовки.

Метод полностью аддитивного формирования слоев (ПАФОС)

Аддитивные технологии -- послойное формирование изделия без удаления лишнего материала.

Печатный рисунок формируется селективно на заготовке из нержавующей тсали. Затем рисунок впессовывеатся в изоляцонный слой на всю толщину проводника, после чего отделяют от временного носителя механическим способом.

Контроль.

В производстве ПП применяются следющие виды контроля:
\begin{enumerate}
	\item Оптический
	\item Электрический
	\item Рентгеновский
\end{enumerate}

Оптический контроль применяют для обнаружения следующих деффектов:
\begin{enumerate}
	\item Проколы, выступы, царапины, вырывы на проводниках;
	\item Неточность размещения контактных площадок и проводников;
	\item Неточность размеров контактных площадок и проводников;
	\item Непараллельность, неровность, подтравливание и нависание краев проводников.
\end{enumerate}

Методы Оптического и визуального контроля

Метод сравнения с эталоном --- сравнение контролируемого изображения с идеальным (эталоном). Отличия классифицируются в соответствии с допусками, определяемыми технологией как существенные (дефекты) или несущественные.

Безэталонный метод контроля --- Эталон изображения отсутсвует. Дефекты в этом случае определяются как нарушение определенных правил которым должна удовлетворять правильная топология.

Почти 70\% всех которольных операций печатных плат составляет котроль внешнего вида.

Достоверность при выполнении контроля внешнего вида человеком составляет около 65\%. Тип контроля: визуальный и оптический.

Визуальный контроль -- контроль органами зерния человека

Оптический контроль -- контроль с использованием оптических средств.

Сама операция как визуального так и оптического контроля для человека утомительная, монотонная и требует напряженного труда.

Пример оптических средств контроля:

Стереоскопическая безокулярная система (стереомикроскоп)

%Еще одна лаба на следующей неделе в среду (или через неделю)

Таким образом актуальной становится автоматизация процесса контроля с использованием системы автоматического оптического контроля (АОИ). Обязательна работа с эталоном и допуск на эталон

При оптическом контроле системы оптического контроля ПП работают под управлением компьютера и снабжены:
\begin{enumerate}
	\item Рабочей платформой с перемещением по X и Y
	\item Фото- или видеооборудование
	\item Цветным монитором с высоким разрешением
	\item Сменными объективами для изменения масштаба изображения
	\item Волоконно-оптической подсветкой контролируемой области
	\item СпециализированныйПО для оптического контроля
\end{enumerate}

Электрический контроль

ПРи электрическом контроле ПП проверяется:
\begin{enumerate}
	\item ЦЕлостность проводников
	\item Наличие короткого замыкания между проводниками
	\item Качество изоляции
\end{enumerate}

Самым технически сложным элементом тестирования ялвяется система контакта с тестируемой платой. Существует енсколько методов эелктричесого контактирования: ручной, "ложе гвоздей", "летающие щупы"

Матричные тестеры:

Контактирование плат с матрицей контактов ("ложей гвоздей" предполагает наличие соединительного устройства с подпружиненными контактами во всех узлах координатной сетки.

Индивидуальность платы учитывают изготовлением маски с перфорациями в местах необходимиого контактирования или изготовлением специального тестового адаптера с размещенными на нем зондами. Во время тестирования все зонды назодятся в контакте с тестовыми точками, и скорость тестирования определяется скоростью переключения ключей. Этот метод обеспечивает высокую производительность, однако требует значительных затрат при переналадке.

Летающие щупы --- Flying Probes Testing System

Оборудование для метода "летающих щупов" имеет несколько головок с приводами по трем осям, на каждой из которых установлен зонд. Головки по программе осуществляют контактирование с с платой, во время которого происходит подача сигнала или измерение. Этот метод обеспечивает простоту переналадки (данные для тестировани получаются из CADICAM-данных), однако производительность кго невысока.

Летающие матрицы

Компромиссом между универсальностью и производительностью является метод "летающих" матриц. В этом случае на каждой каретке размещается матрица щупов, при этом каждыйщуп имеет независимый привод по оси.

Контактирующие зонды.

Контактирующие зонды
Poga-pin (подпружиненны	 контакт)
Существует множество типов пого-пинов (пружинныхх контактов) для тестеров, оличающихся по размерам, форме или рельефу контактной поверхности и т д.

Программное обеспечение

Известно, что для электрическго контрля печатной платы необходимо провести два типа тестов: тест на целостность и тест на разобщенность цепей.

Количество тестирований в для теста на целостность равно $N k$

Для теста на разобщенность -- $N^2/2$.
где $N$ - число тестируемых цепей, $k$ - число звуньев цепей (разветвлений).

Как видно из выкладок, в общем случае для теста на разобщенность требуется значительно больше время тестирования, поэтому важной частью систем тестирования является программное обеспечение, позволяющее оптимизировать тестовую программ.

Исходные данные

Чтобы протестировать печатную плату, необходимо иметь информацию, которая показывает, как печатная плата разведена (список цепей, информация о близко расположенных цепях и т. д.).
Возможно трипути создания списка цепей:
\begin{enumerate}
	\item Список цепей выбирается с использоваинеием эталонной платы.
	\item Список цепей генерируется исходя из данных gerbegr файлов
	\item СПисок цепей заимствуется из CAD-данных.

\end{enumerate}

РЕнтгеновский контроль применяют:
Для контроля качества совмещения внутренних слоев.

Для контроля качества металлизации в слоях

Упрощенная схема системы рентгеновского контроля:
Интенсивность излучения, падающего на детектор, обратно пропорциональна величине поглощения рентгеновский лучей. Чем тоньше объект и чем меньше атомный вес материала, тем светлее будет изображение на детекторе.

Определение предела прочности прирастяжении и удлинения медного покрытия

Определение прочности на расяжение и удлинения медного покрытия в процентах

ОБъект испытаний -- прямоугольные полоски из каольванически осажденной меди (или в виде собачьей кости") размерами 13х150 мм, толщина меди 50 - 100 мкм. Тестирование проводят на 10 образцах.

Подготовка образцов.

Наочищенную пластину из нержавеющей стали получают защитный рисунок из негативного фоторезиста, на который в дальнейшем осаждается гальваническая медь при плотности тока, идентичной плтности тока при изготовлении ПП. затем с пластины осторожно снимают медные образцы и разбраковывают, подбирая для испытаний образцы без каких-либо дефектов.

Преде испытаниями необходимо провести термоорбработку образцов при температуре (125+-5) градусов в течение 4-6 ч, затем образцы охлаждают до комнатной температуры. Использование нержавеющей стали обуславливается необходимостью в токопроводящем электроде.

Проведение испытания.

Образец помещают в зажимное приспособление измерителя растяжения и прикладывают усилие, обеспечивающее постоянную скорость растяжения (0.5 мм/мин). испытание останавливается при разрушении образца.

Перед испытанием измеряется масса образца. После испытания -- длинна (образец соединяется оп месту разрыва). Для измерений мрименяют прецизионный микрометр.

Определение прочности на отслаивание првоодников

Определяем величину адгезии проводников к материалу основания ПП при нормалььных атмосферных условиях.

Образцы: прямые проводники шириной не менее 0.8 мм, длиной -- более 75 мм.

Перед испытаниям с помощью скальпеля один конец проводника аккуратно отделяют от основания платы на длину около 10мм.

Проведение испытаний:

Отеделенный конец проводника захватывают по всей ширине зажимом тестера, а затем перпендикулярно поверхности платы прикладывают постепенно увеличивающееся усилие, чтобы проводник начал отделяться от основания печатной платы с постоянной скоростью. Испытание продолжают до тех пор, пока отслоившийся участок проводника не достигнет длины 25 мм.

За прочность на отслаивание проводника от основания печатной платы принимают минимальное значение силы на единицу ширины, необходимое для отслаивания проводника, полученное при спытании по крайней мере четырех проводников.

Определение прочтности на отрыв контактных площадок.

Контроль адгезии контактных площадок к основанию печатной лпаты после операции пайки. За прочность на отрыв контактных площадок принимают силу, перпендикулярную поверхности печатной платы, необходимую для отделения контактной площадки от материала основания.

Испытания проводят на круглых контакных площадках, отделенных от примыкающих проводников. В отверстия, асположеные в центре контактных площадок, впаивают отрезки проволочек.

Перед испытаниями провдят несколько циклов перепайки.

Для отрыва используют тестер, способный обеспечить опстепенно увеличивающееся усилие со скростью, не превышающей 50 мм/мин, под прямым углом к поврехности образца.

Усилие прикладывают до полного отрыва контактной площадки. За прочтность на отрыв контактных площадок принимают минимальный результат, полученный при отрыве десяти испытанных контактных площадок от материала основания.

Определение прочности на вырыв покрытий из сквазного металлизированного отверстия.

Оценивается способность сквазных метллизированных отверстий без контактных площадок выдерживать неоднократные перепайки.

Объекты: металлизированные отверстия без контактных площадок.

Перед испытаниями в отверстия впаливают облеженные отрезки проволоки, повторяя эту операцию несколько раз. Конец проволоки должен выступать на противоложной стороне обрзац на расстояние не менее 1.5 мм.
%что-то еще

Метод липкой ленты: Контроль адгезии паяльной маски и Контроль адгезии металлического покрытия

Для испытания исопользуется самоклеящаяся липкая лента шириной 1.3 см, чувствительная к давлению, имеющая прочность прилипания от 44 до 66 Н/100мм.

котроля адгезии паяльной маски, нанесенной на поверзность печатной платы сверху оплавяемых и неоплавляемых покрытий. Контроль адгезии проводят как перед пайкой, так и после пайки печатных плат. Тест купон должен быть строго опредленногой формы - в соот с ГОСТ Р 55693-2013.

Для изготовления тест-купона берут фольгированные материал с толщиной фольки 35 мкм.

Отрезок липкой ленты длиной 50 мм прижимают к поверхнсоти тест-купона без воздушных пузырьков. Липую ленту следует располагать поерек образца, удаление пленки проводят быстрым отрывом ее под прямым углом к поверхно
\end{document}



