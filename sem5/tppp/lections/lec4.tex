\documentclass{article}
\begin{document}
1) Поверхностное $R_s$ и удельное сопротивление $\rho_s$
2) Объемное и удельное обхемное сопротивление $\rho_v$
Удельные сопротивления --- фундаментальные свйоства диэлектриков. Данные параметры характеризуют величину тока утечки
Сквозной ток утечки --- ток, проходящий через электрическую изоляцию в установавшемся режиме ( после достаточно продолжительно промежутка времени после приложения напряжения)

Поверхностным называется сопротивление, которым обладает диэлектрик при протекании постоянного тока по его поверхности

Удельное поверхностное сопротивление $\rho_s$ численно равно сопротивлению между противолоположными сторонами поверхности квадрата с площадью $1 м^2$ току, проходящему по поверхности через две противоположные стороны этого квадрата; измеряется в Ом.

Под объемным понимается сопротивление, которым обладает диэлектрик при протекании постоянного тока через его объем.
Удельное объемное сопротивление численно равно сопротивлению куба с ребром в 1 м (мысленно выделенного из исследуемого материала) при протекании тока через две противоположные грани куба; Измеряется в $Ом * м$

Удельное сопротивление диэлектриков зависит от природы диэлектрика ( его диэлектрической проницаемости, пористости, твердости, гигроскопичности и т. д.), наличий загрязнений и различных дефектов поверхности, температуры, влажности, величины приложенного напряжения и времени воздействия.

Относительная диэлектрическая проницаемость ($\varepsilon$) --- отражжает степень влияния среды на действие элктрического поля (уменьшение действия) - число, показывающее во сколько раз кулоновская сила взаимодействия точечных в вакууме больше такой же силы в данной среде.
Объемное и поверхностное сопротивления определяются экспериментально.
На металлические электроды подается напряжение постоянного источника, после чего делается выдержка в течение 1-3 минут, пока не закончатся процессы поляризации. По их окончании между электроадми измерают ток $I$ и напряжение $U$ с помощью микроамперметра и вольтметра.

3) Электрическая прочность изоляции $E_пр$ --- минимальная напряженность поля при которокой наступает электрический пробой --- образуется искра или дука ил в цепи появляется электрический ток.

Электрическая прочность изоляции определяется как напряжение, приходящееся на 1 мм толщины изоляции и змеряется в $\frac {В} {мм}$
3) Электрическая прочность изоляции $E_пр$ --- минимальная напряженность поля при которокой наступает электрический пробой --- образуется искра или дука ил в цепи появляется электрический ток.

Электрическая прочность изоляции определяется как напряжение, приходящееся на 1 мм толщины изоляции и змеряется в $\frac {В} {мм}$

4) Тангенс угла диэлектрических потерь $\tan{\delta}$
Диэлектрическимми потерями называют энергию, рассеимваемую в электроизоляционном материале под воздействием на него электрического поля.
Способность диэлектрика рассеивать энергию в электрическом поле характеризуют тангенсом угла диэлектрических потерью
При испытании диэлектрик рассматривается как диэлектрик конденсатора, у которого измеряется емкость и угол $\delta$. дополняющий до 90 градусов угол сдвига фаз между током и напряжением в умкостной цепи. Этот Угол называется углом диэлектричеких потерь.

5) Механическая прочность - предел прочности пр растяжении и изгибе, который зависит от типа используемой смолы и снижается при повышении температуры.
6) Стабильность линейных расмером при повышении температуры --- зависит отплоустойчивости и теплопроводности армирующего материала, а также от температуры стеклования смолы.
Теммпература стеклования $T_g$ --- температура при которой полимер при охлаждении переходит из высокоэлатсического или вязкотекучего в стеклообразное состояние.
При нагреве, например пайке, происходит значительное расширение полимеров по оси $Z$. Привысокой $T_g$ процесс расширения сдвигается в область более высоких температур.
7) Теплоустойчивость --- устойчивость к воздействию теплового удара
8) устойчивость к агрессивным средам --- устойчивость к щелочам, кислотам, растворителям и пр., что определяет технологические возможности процесса обработки материала.
Напрмер, эпоксидные смолы устойчивы к щелоам, а полиэфирные -- нет. Рпи этом эпоксидные смолы не выдерживают воздействия сильных кислот, в отличие от полиэфирных.
9) Горючесть материала --- способность материала к гашению пламени после воспрламеннения (допускается примерение в ЭА только самозатухающих диэлектрикоа).

*Стандарты категорий UL94 (научно-исследовательских лабораторий страховых компаний) для пластиков
*ГОСТ 302440-94 "Материалы строительные, методы испытания на горючесть"

10) Водопоглощение --- это способность материала впитывать в себя воду и удерживать ее. Величина водопоглощения определяется разностью веса образца, насыщенного водой, и веса сухого образца. Различают объемное водопоглощение $в_об$, когда указанная разность весов отнесена к объему образца, и весовое водопоглощение $В_вес$, когда эта разность отнесена к весу сухого образца

11) Температурный кэффициент линейного расширения (ТКЛР)
В твердых телах тепловое расширение связано с несимметричностью тепловых колебаний атомов, благодаря чему при изменении температуры меняется среднее равновесное положение атомов друго относсительно друга (межатомные расстояния). И это приводит к изменению объема твердого тела.
Степень изменения линейного размера телпа при изменении температуры характеризуется линейным коэффициентом теплового расширеня $\alpha$, вырадающим относитлеьное изменение длины тела при изменении его температуры на один градус

Если обозначить как $l_0$ изначальную длину тела при стартовой температуре $T$, $\Delta l$ - его удлинение при увеличени температуры тела на $\Delta l$, то коэффициент линейного расширения будет определяться по формуле:
$\alpha = \frac {1 \Delta l} {l_0 \Delta T}$
медь - $16,6 * 10^{-6}$
стеклостектолит FR-4 - $60-80 \frac {1}[K}$
\end{document}
