\documentclass{article}
\usepackage[utf8]{inputenc}
\usepackage[english,russian]{babel}
\begin{document}
Под множеством понимают совокупность объектов любой природы, называемых элементами данного множества, обладающих каким-либо общим для множества свойств.

Множество может быть задано: перечислением всех его элементов; указанием на характерное свойство всех элементов данного множества;

Число элементов множества называют его мощностью.

Если число элементов конечное то и множество называют конечным.

Множество которое не содержит ни одного элемента называют пустым. %Здесь должен быть значек пустого множества

Если порядок элементов в множестве важен, то такое множество называется упорядоченным, а последовательность из $n$ элементов называют $n$ строкой.

Множества называют равнымми если они состоят из одних и тех же элементов. Если все элементы множества $X$ принадлежат множеству $Y$, то говорят что множество $X$ является подмножеством множества $Y$. %здесь должны быть обозначения (см тетрадь

Пересечением множеств $X$ и $Y$ называют множество состоящее из общих для этих множеств элементов.


Объединение множеств приводит к образованию нового множества, которое получается из элементов принадлежащих хотя бы одному из множеств $X$ или $Y$.

$X / Y$ разность множеств $X$ и $Y$ состоит из всех элементов не принадлежащих множеству $Y$ но приндалежащих множеству $X$.

Разбиением множества $X$ называется такое множество множеств из $X_i$ при котором выполнены следующие условия: %Здесь должен быть список условий в мат записи

Отношения между множествами:
\begin{itemize}
	\item Эквивалентности
	\item Порядка
	\item Доминирования
\end{itemize}
Элементы называются эквивалентными если любой из этих элементов можно заменить другим.

Свойства эквивалентности:
\begin{enumerate}
	\item Рефлексивность $X \equiv X$ --- истинно
	\item Симметричность: Если $X \equiv Y$, то $Y \equiv X$
	\item Транзитивность: Если $X \equiv Y$ и $Y \equiv Z$, то $X \equiv Z$
\end{enumerate}
Свойства отношения порядка:
\begin{itemize}
	\item Строгого:
		\begin{enumerate}
			\item Антирефлексивность $x < y$ -- ложно
			\item Несеммитричность $x < y$ и $y < z$, то $x < z$
			\item Транзитивность: Если $x < y$ и $y < z$, то $x < z$
		\end{enumerate}
	\item Нестрогого:
	\begin{enumerate}
		\item Рефлексивность $x \le y$ -- истинно
		\item Антисеммитричность Если $x < y$, $y \le z$, то $y = x$
		\item Транзитивность Если $x \le y$ и $y \le z$, то $x \le z$
	\end{enumerate}
\end{itemize}
Отношения доминирования (>>) --- Говорят что элемент $x$ доминирует над элементом $y$ того же множества если $x$ в чем-то превосходит $y$

Это отношение не обладает свойством транзитивности.

Под графом понимают совокупность непустого множества $X$ и множества $U$ представляющее собой множество упорядоченных пар ${X_i;X_j}$

Элементы множеств $X$ и $U$ называются соответственного вершинами и ребрами графа. ($G(X,U)$)
Способы задания графа:
\begin{enumerate}
	\item Геометрический
	\item Аналитический
	\item Матричный
\end{enumerate}

Две вершины называются смежными, если они определяют ребро, два ребра смежные если они имеют общую вершину.

Вершина инцидентна ребру если она является началом или концом этого ребра

Ребро инцидентно вершине если эта вершина ему принадлежит.

Чило дуг инцидентных некоторой вершине называют степенью ее инцидентности $\rho(x_i)$.

$\rho(x_1)=4$
$\rho(x_2)=2$
$\rho(x_3)=2$
$\rho(x_4)=2$
%Тут должна быть формула соответствия суммы степеней инцидентности и числа ребер

Следствие --- число вершин с нечетной степенью инцидентности всегда четное
Вершину не инцидентную ребру называют изолированной, граф состоящий только из изолированных вершин называют нуль-графом.
Граф все вершины которого попарно смежны называют полным графом.
Пусть задан полный граф содержащий $n$ вершин.
$\rho(x_i)=n-1$
%Сумма
Следовательно число ребер полного графа:
$U = n(n-1)/2$
Граф в котором существует хотя бы одна пара вершин соединенной несколькими ребрами называется мультиграфами.

Раскраска графа --- это разбиение множества его вершин на подмножества, такое что каждое из подмножеств не содержит смежных вершин

Хроматичность --- наименьшее число множеств на которое можно раскрасить граф.

Последовательность заданная парами вершин называется маршрутом графа.

Цепь --- маршрут в котором все ребра различны.

Цикл --- цепь в которой совпадает начальная и конечная вершина.

Компонент связности --- связанная часть графа.
\end{document}
