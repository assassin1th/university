\documentclass{article}
\usepackage{amsmath}
\usepackage{mathtext}
\usepackage[english,russian]{babel}
\usepackage[T2A]{fontenc}
\usepackage[utf8]{inputenc}
\begin{document}
Вершинами в данной схеме являются узлы, а ребрами графа Токопроводящие дорожки соединяющие эти узлы. Для получения топологических уравнений определяют число узлов схемы, затем разбивают весь граф
на 2 подмножества хорд и ветвей. Подмножество ветвей должно состоять из $\beta - 1$ ветви необразующих ни 1 замкнутый контур, остальные ребра графа назовем хордами. Тогда компонентные топологические уравнения будут составляться:
$$
U_x + M U_b = 0
$$
$$
I_B - M^T I_X = 0
$$
M - матрица ,формируемая по павилу: усли при подключении к множеству ветвей данная хорда образует замкнутый контур, то компомнент матрица равен 1 или -1. Если замкнутый контукр с данной ветвью не образуется то компонент матрицы равен 0. Выбор знака осуществляется в зависимости от того -- совпадает ли направление подключаемой хорды, с подключаемой ветвью.

Подмножество ветвей: $C_1 C_2 C_3$
Подмножество вершин: $J, R_1, R_2, R_3, R_4$
$$
\begin{pmatrix}
	-1 & 0 & 0\\
	0 & -1 & 0\\
	0 & 0 & -1\\
	-1 & 1 & 1\\
	1 & 0 & 0
\end{pmatrix}
$$
$$
\begin{cases}
	U_{C_1} - U_{C_2} = 0\\
	U_{R_2} - U_{C_2} = 0\\
	U_{R_3} - U_{C_2} = 0\\
	U_{R_4} - U_{C_1} + U_{C_2} + U_{C_3} = 0\\
	U_J + U_{C_1} = 0
\end{cases}
$$
$$
\begin {cases}
	I_{C_1} + I_{R_1} + I_{R_4} - J = 0\\
	I_{C_2} + I_{R_2} + I_{R_4} = 0\\
	I_{C_3} + I_{R_3} - I_{R_4} = 0
\end{cases}
$$
Матрица смежности вершин - квадратная матрица формируемая по правилу: $A_{ij} = 1$ если $i$ и $j$ элементы смежные и $A_{ij} = 0$ в противном случае
$$
\begin{pmatrix}
	0 & 1 & 1 & 1 & 0\\
	1 & 0 & 1 & 0 & 0\\
	1 & 1 & 0 & 1 & 1\\
	1 & 0 & 1 & 0 & 0\\
	0 & 0 & 1 & 0 & 0
\end{pmatrix}
$$
Матрица смежности ребер состоит из элементов $W_{ij} = 1$ в случае смежности $i$ и $j$ ребра и $W_{ij} = 0$ в противном случае
$$
\begin{pmatrix}
0 & 1 &1 & 0 & 0 & 0\\
1 & 0 & 1 & 1 & 1 & 1\\
1 & 1 & 0 & 0 & 1 & 0\\
1 & 1 & 0 & 0 & 1 & 1\\
0 & 1 & 1 & 1 & 0 & 1\\
0 & 1 & 0 & 1 & 1 & 0
\end{pmatrix}
$$

Матрица инцидентности S - прямоугольная $n * m$ (n -- число вершин, m -- число ребер) $S_{ij} = 1$ если вершина $i$ инцидентная ребру $j$

$$
\begin{pmatrix}
1 & 1 & 1 & 0 & 0 & 0\\
1 & 0 & 0 & 1 & 0 & 0\\
0 & 1 & 0 & 1 & 1 & 1\\
0 & 0& 1 & 0 & 1 & 0\\
0 & 0 & 0 & 0 & 0 & 1
\end{pmatrix}
$$

Матрица весовых соотношений - квадратная матрица, в которой элемент равен количеству связей между вершинами графа.
Матрица длин -
\begin{enumerate}
	\item $d_{ij} = \sqrt{(x_i - x_j) ^ 2 + (y_i - y_j)^2}$
	\item $d_{ij} = (x_i - x_j) + (y_i - y_j)$
	\item $d_{ij} = (x_i - x_j)^{2k} + (y_i - y_j)^{2k}$
\end{enumerate}
Формальное представление о схеме цепи. Ппонятие о списке цепи
$$
\begin{pmatrix}
	x_0;x_1 & c_{01}; c_{11}\\
	x_0;x_1;x_3; & c_{02}; C_{12}; C_{32}\\
	x_1;x_2;x_3; & c_{13}; c_{21}; c_{31}\\
	x_2;x_3;x_4 & c_{22}; c_{33}; c_{41}\\
	x_0;x_2 & c_{03}; c_{23}\\
	x_0;x_4 & c_{04}; c_{42}
\end{pmatrix}
$$
\end{document}
